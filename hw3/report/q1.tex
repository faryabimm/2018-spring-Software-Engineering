%%%%%%%%%%%%%%%%%%%%%%%%%%%%%%%%%%%%%%%%%%%%%%%%%%%%%%%%%%%%%%%%%%%%%
%%%%%%%%%%%                   PROBLEM 1                   %%%%%%%%%%%
%%%%%%%%%%%%%%%%%%%%%%%%%%%%%%%%%%%%%%%%%%%%%%%%%%%%%%%%%%%%%%%%%%%%%

\section{پرسش اول}
موارد خواسته‌شده در مورد مهندسی نیازمندی امنیتی را شرح می‌دهیم:

\subsection{مدل گراف حمله\lr{(Attack Graph)}}
دکتر 
\lr{Cynthia Phillips}
به همراه یکی از همکارانشان در سال ۱۹۹۸ مدلی گرافی را برای بررسی نقاط ضعف در شبکه‌های کامپیوتری معرفی کردند. این مدل مکانیزمی برای بررسی حملات و نسبت دادن آنها به ماشین‌های داخل شبکه و مهاجمین ارائه می‌کند.

این مدل بر اساس طرح‌ها و قدمات حملات شناخته‌شده‌ی قبلی در سیستم‌های دیگر، پیکربندی شبکه‌ی سیستم فعلی و بررسی و زیر نظر گرفتن شبکه توسعه می‌یابد.
\cite{attack-graph}

گراف حمله بخش مهمی از تحلیل نفوذ پذیری در سیستم‌های تحت شبکه است و
از آنجایی که تهیه‌ی چنین مدلی به صورت دستی آسان نیست، روشی اتوماتیک وغیر رسمی برای ایجاد آن در طراحی‌های iterative پیشنهاد شده است. در این روش، اهداف سیستم، اهداف متهاجمین احتمالی و مسیر‌های تهاجم با  احتمال موفقیت بالا را مدل سازی کرده و مورد بررسی  و مستندسازی قرار می‌دهد.
\cite{auto-attack-graph}

\subsection{UMLsec}

UMLsec
(با secureUML اشتباه گرفته نشود)
افزونه‌ای به زبان مدلسازی UML است که از طریق افزودن مجموعه‌ای از 
stereotype ها و tag ها و constraint ها به زبان UML
سعی در گنجاندن مفاهیمی و اطلاعات مربوط امنیتی در دل نمودار‌های UML مانند
\lr{State Diagram}
،
\lr{Sequence Diagram}
و 
\lr{Activity Diagram}
دارد.
هدف اصلی این افزونه کپسوله کردن مفاهیم امنیتی و رخداد‌های پرتکرار در حوزه‌ی امنیت و ارائه‌ی آن به توسعه‌دهندگان در قالب مجموعه‌ای منسجم است.

به وسیله‌ی طراحی غنی شده با این افزونه می‌توان تهدید‌های امنیتی خاص را مدل کرده، بررسی کرد و تاثیر آنها را بر امنیت و کارایی کل سیستم در زمان اجرای این تحریک‌ها مورد مطالعه قرار داد.
\cite{umlsec}

\subsection{مورد سوء استفاده\lr{(Abuse Case)}}

مدل سوء استفاده مدلی است که از نوتیشن UML با نماد‌های برعکس شده استفاده می‌کند. هدف این کار نشان دادن functionality است که از سیستم انتظار 
\textbf{نمی‌رود.}

این مدل بر پایه‌ی زبان مدل‌سازی UML بنا شده است.

\lr{Use Case}
ها موارد کاربرد مورد انتظار و مثبت سیستم هستند و کارایی سیتم نرم‌افزاری را نشان می‌دهند. در مقابل
\lr{Abuse Case}
ها جزئیات تعاملاتی با سیستم را نشان می‌دهند که برای سیستم، محیط آن یا کاربران آن مخاطراتی را به همراه خواهند داشت.

یک
\lr{Abuse Case}
مراحلی را نشان می‌دهد که از امکانات مجاز یک سیستم برای اتمام سوء استفاده طی می‌شود.

\cite{umlsec}
