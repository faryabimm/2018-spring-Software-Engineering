%%%%%%%%%%%%%%%%%%%%%%%%%%%%%%%%%%%%%%%%%%%%%%%%%%%%%%%%%%%%%%%%%%%%%
%%%%%%%%%%%                   PROBLEM 2                   %%%%%%%%%%%
%%%%%%%%%%%%%%%%%%%%%%%%%%%%%%%%%%%%%%%%%%%%%%%%%%%%%%%%%%%%%%%%%%%%%

\section{پرسش دوم}

برای هر کدام از حوزه‌های خواسته شده دو الگوی طراحی را شرح می‌دهیم:

\subsection{پردازش ابری}

\subsubsection{اتوماسیون مدیریتی}
\begin{itemize}
\item \textbf{نام} \newline
اتوماسیون مدیریتی یا \lr{Automated Administration}
\item \textbf{توضیحات} \newline
چگونه باید وظایف مدیریتی رایج را به صورت مداوم و اتوماتیک در پاسخ به رخداد‌های از پیش تعیین شده به اجرا در آورد.
\item \textbf{مشکل} \newline
منابع در سیستم‌های فناوری اطلاعات درستخوش تعداد زیادی فعالیت مدیریتی تکراری و مداوم می‌شوند. سپردن این فعالیت‌ها به عامل انسانی احتمال بروز خطا و مشکل در عملیات را افزایش می‌دهد.
\item \textbf{راه‌حل} \newline
مراحل کار فعالیت‌های مدیریتی مناسب خودکار‌سازی است. برای این منظور script هایی ایجاد شده و در یک سیستم خودکار اجرا کننده قرار داده می‌شوند.
\item \textbf{کاربرد} \newline
یک موتور اتوماسیون هوشمند در کنار سیستم اطلاعاتی قرار می‌گیرد تا فعالیت‌های تکراری و مداوم مدیریتی را متناسب با شرایط اجرا کند.
\item \textbf{منبع} \newline
\url{http://cloudpatterns.org/design_patterns/automated_administration}
\end{itemize}

\subsubsection{پرداخت به میزان مصرف}
\begin{itemize}
\item \textbf{نام} \newline
پرداخت به میزان مصرف \lr{Pay as You Go}
\item \textbf{توضیحات} \newline
چگونه باید از یک مشتری، متناسب با هزینه‌ی مصرف منابع او  هزینه دریافت کرد.
\item \textbf{مشکل} \newline
خریداری یا اجاره‌ی یک منبع IT ممکن است با هزینه‌هایی همراه باشد که با میزان مصرف و استفاده‌ی واقعی از آن منبع  متفاوت باشد.
\item \textbf{راه‌حل} \newline
سیستمی معرفی و استفاده می‌شود که میزان مصرف را اندازه‌گیری کرده و متناسب با آن تخصیص هزینه‌ی لازم را به کاربر انجام دهد.
\item \textbf{کاربرد} \newline
سیستم نظارت در زمان اجرا و مصرف و سیستم محاسبه‌ی هزینه‌ها به صورت آنی مصرف کاربر را زیر نظر گرفته و هزینه‌های لازم را صادر می‌کنند.
\item \textbf{منبع} \newline
\url{http://cloudpatterns.org/design_patterns/pay_as_you_go}
\end{itemize}


\subsection{تحمل اشکال\lr{(Fault Tolerant)}}

\subsubsection{من زنده‌ام}
\begin{itemize}
\item \textbf{نام} \newline
من زنده‌ام یا \lr{I am alive}
\item \textbf{توضیحات} \newline
سیستمی داریم که در آن هزینه‌ی ارتباط با واحد نظارت قابل تحمل و پهنای باند مورد نیاز برای ارتباط با سیستم نظارت به وضوح کمتر از حداکثر پهنای باند است. در این سیستم زمان یک تعامل با واحد نظارت محدود یا شناخته شده است.
\item \textbf{مشکل} \newline
ممکن است یک سرویس یا سیستم حین کار خاموش شده یا از دور خارج شود. نیاز به وجود مکانیزمی برای تشخیص سریع و اقدام برای این مورد حس می‌شود.
\item \textbf{راه‌حل} \newline
سیستم نظارت به صورت دوره‌ای و منظم سیستم مورد نظارت را چک کرده و از صحت عملکرد آن مطلع می‌شود. به این ترتیب این سیستم اطلاعات خود را از الگو‌های خطایی سیستم مورد نظارت به تدریج کامل می‌کند. در صورت بروز خطا، سیستم پشتیبانی به سرعت درگیر و فعال می‌شود.
\item \textbf{کاربرد} \newline
سیستم چک کردن خودکار زنده‌بودن در کنار و به عنوان عضوی از سیستم نظارت فعال شده و به کار می‌پردازد.
\item \textbf{منبع} \newline
\url{https://pdfs.semanticscholar.org/117c/c559abe000542128da6f24b0e31319db1eca.pdf}
\end{itemize}

\subsubsection{افزونگی}
\begin{itemize}
\item \textbf{نام} \newline
افزونگی یا \lr{‌Redundancy}
\item \textbf{توضیحات} \newline
ممکن است در یک سیستم نرم‌افزاری بنا به دلایلی، سیستم از کار بیفتد یا اطلاعات از بین برود.
\item \textbf{مشکل} \newline
از دست رفتن اطلاعات یا سایر منابع به علت بروز خطا ممکن است در یک سیتم نرم‌افزاری پیش بیاید.
\item \textbf{راه‌حل} \newline
با نگهداری نسخه‌های مختلفی از یک منبع مانند پایگاه داده یا منابع فیزیکی و سخت افزاری و یا یم سرویس نرم‌افزاری می‌توان در صورت از کار افتادن نسخه‌ی قبلی، نسخه‌ی دیگری را جایگزین آن کرد.
\item \textbf{کاربرد} \newline
سیستمی برای نظارت و مدیریت نسخ مختلف یک منبع و همچنین تقسیم کار بین آنها ایجاد شده و مدیریت این منابع را بر عهده می‌گیرد.
\item \textbf{منبع} \newline
\url{https://pdfs.semanticscholar.org/9341/0c000b781ba73210aa0d2958195a6193e64b.pdf}
\end{itemize}

\subsection{تست مکانیزه\lr{(Automated Testing)}}


\subsubsection{صفحه شیء}
\begin{itemize}
\item \textbf{نام} \newline
صفحه شی یا \lr{‌Page-Object}
\item \textbf{توضیحات} \newline
برای تست عملکرد در رابط گرافیکی لازم است اطلاعات مختلفی مانند تعاملات با صفحه و آنچه در صفحه مشاهده می‌شود دخیل شود.
\item \textbf{مشکل} \newline
تست رابط گرافیکی و عملکرد سیستم در آن ممکن است به دلیل ماهیت تعاملی آن با انسان بسیار پیچیده و سخت باشد
\item \textbf{راه‌حل} \newline
با نگهداری تمامی اطلاعات تست مانند تعاملات با صفحه و وردوی‌ها و تغییرات رابط گرافیکی در یک شی واحد می‌توان بررسی تغییرات و انجام تست را بسیار ساده تر کرد.
\item \textbf{کاربرد} \newline
سیستمی برای تقسیم بندی جزئی تمامی بخش‌های تعامل کاربر با محصول مانند صفحعه‌ی نمایش و موقعیت اشاره گر ایجاد شده که اطلاعات را در قالب یک سری frame و cell ذخیره می‌کند. این اطلاعات برای انجام تست و تهیه‌ی مدل‌های تست مورد استفاده قرار می‌گیرد.
\item \textbf{منبع} \newline
\url{https://www.automatetheplanet.com/page-object-pattern/}
\end{itemize}


\subsubsection{صورت}
\begin{itemize}
\item \textbf{نام} \newline
صورت یا \lr{‌Facet}
\item \textbf{توضیحات} \newline
برای تست عملکرد یک سیستم گاهی لازم است تستی مرکب شامل تست روی چندین صفحه انجام شود.
\item \textbf{مشکل} \newline
تست یک عملکرد سیستم نرم‌افزاری ممکن است به علت ماهیت مرکب و طولانی آن شامل چندین صفحه‌ی کاربری باشد و باعث دشواری آزمون شود.
\item \textbf{راه‌حل} \newline
با نگهداری اطلاعات چندین تست در یک شی facet می‌توان تست را روی این شی انجام داد. به این ترتیب هر facet شامل اطلاعات چند page-object خواهد بود و مکانیزمی برای اجرای یک باره‌ی کل تست ارائه خواهد شد.
\item \textbf{کاربرد} \newline
اطلاعات مرکب تست توسط سیستم تست به صورت خودکار ضبط و بسته بندی خواهد شد و اجرای تست مرکب در دفعات آتی همانند اجرای تست ساده خواهد بود. 
\item \textbf{منبع} \newline
\url{https://www.automatetheplanet.com/facade-design-pattern/}
\end{itemize}
