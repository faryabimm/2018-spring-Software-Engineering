%%%%%%%%%%%%%%%%%%%%%%%%%%%%%%%%%%%%%%%%%%%%%%%%%%%%%%%%%%%%%%%%%%%%%
%%%%%%%%%%%                   PROBLEM 4                   %%%%%%%%%%%
%%%%%%%%%%%%%%%%%%%%%%%%%%%%%%%%%%%%%%%%%%%%%%%%%%%%%%%%%%%%%%%%%%%%%

\section{پرسش چهارم}

سطوح مختلف جفت شدگی 
\LTRfootnote{Coupling}
و پیوستگی
\LTRfootnote{Coheision}
را به همراه مثال تشریح می‌کنیم:

\subsection{سطوح مخالفت جفت شدگی}
\begin{enumerate}
\item \textbf{بالاترین سطح جفت‌شدگی} \newline
\begin{enumerate}
\item \textbf{جفت‌شدگی در محتوا\LTRfootnote{Content Coupling}} \newline
ماژول A به محتوا و بخش‌های private ماژول B دسترسی پیدا می‌کند. این کار ممکن است از طریق اجرای یک دستور  GOTO در ماژول A اتفاق بیفتد. این سطح از جفت شدگی به هیچ عنوان نباید اجازه داده شود
\end{enumerate}
\item \textbf{جفت‌شدگی بالا} \newline
\begin{enumerate}
\item \textbf{جفت‌شدگی عادی\LTRfootnote{Common Coupling}} \newline
دو ماژول یا بیشتر که به داده‌ی Global مشترکی دسترسی دارند، با یکدگیر جفت شدگی عادی دارند. این جفت شدگی مورد تایید نیست ولی شاید اجتناب ناپذیر باشد.
\item \textbf{جفت‌شدگی خارجی\LTRfootnote{External Coupling}} \newline
چند ماژول که به یک دستگاه IO دسترسی دارند، با  یکدیگر جفت شدگی خارجی دارند. این جفت شدگی مورد تایید نیست ولی شاید اجتناب ناپذیر باشد.
\end{enumerate}
\item \textbf{جفت‌شدگی میانه} \newline
\begin{enumerate}
\item \textbf{جفت‌شدگی در کنترل\LTRfootnote{Control Coupling}} \newline
اگر ماژول A با پاس دادن اطلاعاتی به ماژول B آن را کنترل کند گوییم ماژول‌های A و B با یکدیگر جفت شدگی در کنترل دارند. مثلا خروجی یک رویه در ماژول A وضعیت را در یک حلقه در ماژول B تعیین کند.

این جفت شدگی قابل قبول است ولی مدل‌های طراحی و پیاده سازی باید به صراحت وجود این کنترل را تایید کنند.
\end{enumerate}
\item \textbf{جفت‌شدگی کم} \newline
\begin{enumerate}
\item \textbf{جفت‌شدگی در داده\LTRfootnote{Data Coupling}} \newline
گوییم دو ماژول جفت شدگی در داده دارند اگر تنها تعاملات این دو در پاس دادن پارامتر‌هایی ساده به رویه‌های یکدیگر باشد. این جفت شدگی تنها جفت شدگی عادی و قابل قبول برای دو ماژول است و جفت‌شدگی‌های دیگر تنها در صورتی که نیاز باشد مورد اسفتاده هستند.
\item \textbf{جفت‌شدگی در داده‌ی پیچیده\LTRfootnote{Stamp Coupling}} \newline
گوییم دو ماژول جفت شدگی در داده دارند اگر تنها تعاملات این دو در پاس دادن پارامتر‌هایی مرکب به رویه‌های یکدیگر باشد.
\end{enumerate}
\item \item \textbf{پایین ترین سطح جفت شدگی} \newline
گوییم دو ماژول پایین ترین سطح جفت شدگی را دارند اگر این دو اصلا هیچ تعاملی با یکدیگر نداشته باشند.
\end{enumerate}

\url{http://pages.cpsc.ucalgary.ca/~eberly/Courses/CPSC333/Lectures/Design/coupling.html}


\subsection{سطوح مختلف پیوستگی}

\begin{enumerate}
\item \textbf{پیوستگی کم} \newline
\begin{enumerate}
\item \textbf{پیوستگی اتفاقی\LTRfootnote{Coincidental Cohesion}} \newline
ماژولی که کار‌هایی را انجام می‌دهد که هیچ ربط واضحی به یکدیگر ندارد پیوستگی اتفاقی دارد. مانند ماژولی که دو وظیفه‌ی تعمیر سیستم و آپلود داده را بر عهده دارد.
چنین ماژول‌هایی باید حذف و با چندین ماژول خاص منظور جایگزین شوند.
\item \textbf{پیوستگی منطقی\LTRfootnote{Logical Cohesion}} \newline
ماژولی پیوستگی منطقی دارد که رویه‌هایی در آن در یک دسته‌ی عمومی قرار داشته باشند و انتخاب جزئی از بین این دسته‌ها از بیرون ماژول با فراخوانی یکی از آنها انجام شود. مثلا ماژولی که یک رویه برای نوشتن در حافظه‌ی جانبی و یک رویه برای نوشتن در دیسک دارد. معمول نیست یک کاربر هر دوی این دستورات را در یک کار خود انجام دهد. 
برای درست کردن این نکته‌ی منفی می‌توان یک رویه در اختیار بیرون قرار داد که با گرفتن یک پارامتر جزئیات عمل را متوجه می‌شود و رویه‌های قبلی را private کرد.
\item \textbf{پیوستگی زمانی\LTRfootnote{Temporal Cohesion}} \newline
ماژولی پیوستگی زمانی دارد که دارای رویه‌هایی باشد که همگی در زمان به یکدیگر مرتبط اند. مثلا ماژولی را در نظر بگیرید که برای خاموش کردن سیستم استفاده می‌شود و رویه‌ی بستن فایل‌های باز و همچنین قطع اتصال شبکه را ارائه می‌دهد. حین خاموش شدن سیستم این دو کار باید در توالی انجام شوند و دادن انتخاب نسبت به انجام آنها به کاربر کار درستی نیست.
برای حل مشکل رویه‌ی خاموش شدن سیستم ارائه می‌شود که این دو رویه را مورد استفاده قرار خواهد داد. این دو نیز private خواهند شد.
\end{enumerate}
\item \textbf{پیوستگی میانه} \newline
\begin{enumerate}
\item \textbf{پیوستگی در فرایند\LTRfootnote{Procedural Cohesion}} \newline
چنین ماژولی دارای رویه‌هایی متفاوت و بعضا نامرتبط است که کنترل در آن از یکی از رویه‌ها به بعدی منتقل می‌شود. مانند آماده سازی پوشه‌های مورد نیاز و بعد از آن تولید نسخه‌ی پشتیبان.

این پیوستگی مورد قبول است و همچنین بهتر از پیوستگی زمانی است. چرا که روند منطقی بین کار‌ها قابل مشاهده است. ولی کماکان دلیل منطقی برای قرار دادن این کار‌ها در یک ماژول وجود ندارد.
\item \textbf{پیوستگی در ارتباط\LTRfootnote{Communicational Cohesion}} \newline
ماژولی در سطح پیوستگی در ارتباط است که رویه‌هایی داشته باشد که همگی به یک داده‌ی ورودی دسترسی دارند یا همگی یک بخش از ساختمان داده را مورد استفاده قرار می‌دهند. مانند رویه‌های پیدا کردن نام کتاب و پیدا کردن تعداد صفحات کتاب.

یا ماژولی که تمام رابط یک داده ساختار را ارائه کند.
\item \textbf{پیوستگی در توالی\LTRfootnote{Sequential Cohesion}} \newline
ماژولی که دارای رویه‌هایی باشد که خروجی یک رویه به عنوان ورودی رویه‌ی بعدی مورد استفاده قرار بگیرد.

مانند شستن لباس‌ها،
خشک کردن لباس‌های شسته شده و در نهایت اتو کردن لباس‌های خشک شده. 
\end{enumerate}
\item \textbf{پیوستگی بالا} \newline
\begin{enumerate}
\item \textbf{پیوستگی در کاربرد\LTRfootnote{Functional Cohesion}} \newline
ماژولی در کاربرد پیوستگی دارد که تنها رویه‌هایی را شامل شود که برای انجام تنها یک وظیفه‌مندی مربوط به مساله طراحی و ایجاد شده اند.

مثلا ماژولی که سینتکس یک ورودی را بررسی و تایید می‌کند یا ماژولی که برای تعیین صندلی در سینما مورد استفاده قرار می‌گیرد
\end{enumerate}
\end{enumerate}

\url{http://pages.cpsc.ucalgary.ca/~eberly/Courses/CPSC333/Lectures/Design/cohesion.html}
