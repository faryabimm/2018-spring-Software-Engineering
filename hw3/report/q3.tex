%%%%%%%%%%%%%%%%%%%%%%%%%%%%%%%%%%%%%%%%%%%%%%%%%%%%%%%%%%%%%%%%%%%%%
%%%%%%%%%%%                   PROBLEM 3                   %%%%%%%%%%%
%%%%%%%%%%%%%%%%%%%%%%%%%%%%%%%%%%%%%%%%%%%%%%%%%%%%%%%%%%%%%%%%%%%%%

\section{پرسش سوم}
روش‌های خواسته‌شده برای مهندسی نیازمندی‌ها در مهندسی نرم‌افزار را شرح می‌دهیم:

\subsection{مهندسی نیازمندی هدف محور\lr{(Goal Oriented)}}
 در مهندسی نیازمندی‌های هدف محور اینکه سیستم برای چه مورد نیاز است و هدف اصلی آن چیست، مبنا و پایه‌ی اصلی استخراج و مهندسی نیازمندی‌هاست.

بسیاری از متودولوژی‌های نرم‌افزاری از زمان‌های گذشته تا به امروز از این روش و معیار برای مهندسی نیازمند‌ی ها در پروژه استفاده می‌کنند.

در این روش ابتدا اهداف از سیستمی که قرار است تولید شود بیان می‌شوند. سپس سیستم فعلی در ابعاد مختلف سازمانی، اجرایی  مورد بررسی دقیق قرار می‌گیرد و میزان توجه سیستم به نیازمندی‌های قبلی مورد بررسی قرار می‌گیرد. در نهایت سیستم جدید با هدف پاسخگویی بهینه به نیاز‌های کاربران مورد توسعه قرار می‌گیرد.

در چنین روشی خیلی اوقات  نیاز به مدلسازی سطح بالاتری نیاز است. به این ترتیب این مدلسازی سطح پایین تر بسیاری از کلیات مانند تعاملات  و نیازمندی‌های آنها را به وضوح توضیح نمی‌دهد.

\url{https://ieeexplore.ieee.org/abstract/document/948567/}

\subsection{مهندسی نیازمندی عامل محور\lr{(Agent Oriented)}}
به تازگی روش جدیدی برای مهندسی نیازمندی‌ها مطرح شده که در آن به جای مطرح کردن اهداف سیستم به عنوان مرجع مدلسازی،‌عامل‌های سیستم، تعاملات بین آنها و سیستم ساخته شده از ارتباطشان مرجع قرار می‌گیرد.

عامل‌ها در چنین سیستمی با محیط، با یکدیگر و با عوامل انسانی تعامل می‌کنند،‌ هوشمندی دارند،‌ به دنبال موقعیت انجام وظایف خود هستند و حتی می‌توان به صورت انتزاعی به آنها روحیات و سطح کارایی نسبت داد. به این صورت می‌توان استخراج نیازمندی‌ها را بسیار سطح بالا و به زبانی بسیار نزدیک تر به زبان انسان انجام داد. چنین کاری انعطاف مدل‌های ایجاد شده و درک آنها را بسیار ساده تر می‌کند و بسیار قدرتمند تر از مهندسی نیازمندی‌های هدف محور است.

\url{https://pdfs.semanticscholar.org/0dcd/d1324db2a6888010332bd747520611f28ab0.pdf}