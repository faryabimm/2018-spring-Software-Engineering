%%%%%%%%%%%%%%%%%%%%%%%%%%%%%%%%%%%%%%%%%%%%%%%%%%%%%%%%%%%%%%%%%%%%%
%%%%%%%%%%%                   PROBLEM 3                   %%%%%%%%%%%
%%%%%%%%%%%%%%%%%%%%%%%%%%%%%%%%%%%%%%%%%%%%%%%%%%%%%%%%%%%%%%%%%%%%%

\section{تفاوت‌های GitHub و GitLab از نظر قابلیت‌های رایگان}\ \\


\begin{itemize}
\item میزان دسترسی و اعتبار: \newline

برنامه GitHub به کاربران خود این دسترسی را می دهد که برای افراد دیگر, تنها اجازه خواندن یا نوشتن در یک مخزن را تعیین نمایند. اما در سوی دیگر در GitLab کاربران این توانایی را دارند که با توجه به نقش هر یک از افراد سطح دسترسی آنها را تعیین نمایند. در GitLab حتی می توان به کسانی که موضوع را دنبال می کنند, بدون اینکه کد اصلی را در اختیار داشته باشند, دسترسی داد. این ویژگی GitLab است که باعث می شود در پروژه ها و تیم های بزرگ کاربردی باشد.

\item امکانات ادغام و تحویل مداوم (CI و CD): \newline

از مهمترین تفاوت های این دو برنامه این موضوع می باشد. GitLab خودش دارای CI است که این امکان را بصورت رایگان در اختیار کاربران قرار می دهد و نیازی به استفاده از سرویس های بیرونی وجود ندارد. CI باعث صرفه جویی در زمان برای تیم های تولیدکننده کد می شود و همچنین یکی از روش های تضمین کیفیت (QA) نیز می باشد. در سوی دیگر در برنامه GitHub از یک سرویس خارجی به منظور ارائه CI مثل Travis CI یا Codeship برای تست و اجرای برنامه استفاده می کند.

\item محیط و تعداد کاربران: \newline
طبعا GitHub فضای بسیار بزرگتری را میان تولیدکنندگان کد پیدا کرده است و جای بزرگتری را اشغال کرده است. محبوبیت زیاد GitHub ناشی از کاربران میلیونی و محیط بسیار فعال آن است. البته با اینکه GitLab کاربران کمتری در حدود صدهزار نفری دارد ولی فعالیت هایی نظیر میزبانی برای فعالیت های جمعی کدزنی و متن باز بودن این برنامه باعث شده است که GitLab نیز طرفداران خود را داشته باشد. در کل فضای بزرگ توسعه دهندگان GitHub محبوبیت بیشتری برای این برنامه ایجاد کرده است.

\item نسخه تجاری GitLab و GitHub: \newline

هر دو برنامه قابلیت های بسیار قوی در محیط تجاری دارند. GitHub محبوبیت بیشتری میان developer ها دارد درحالی که محبوبیت GitLab بخاطر ویژگی های قوی و متفاوت نسخه تجاری آن, در میان  گروه های ایجاد کد بزرگتر طرفدار بیشتری پیدا کرده است. قیمت های این دو برنامه نیز با هم متفاوت است و نسخه تجاری GitLab قیمت مناسب تری نسبت به نسخه تجاری GitHub دارد.

\item رابط کاربری: \newline

از جهت رابط کاربری, برنامه GitLab از GitHub دارای مزیت بیشتری است زیرا کاربر را قادر می سازد تا در یک صفحه لیستی از پروژه ها و آخرین پروژه ها و کاربران و آخرین کاربران و گروه ها و آمار پروژه را داشته باشد.

\end{itemize}
















