%%%%%%%%%%%%%%%%%%%%%%%%%%%%%%%%%%%%%%%%%%%%%%%%%%%%%%%%%%%%%%%%%%%%%
%%%%%%%%%%%                   PROBLEM 2                   %%%%%%%%%%%
%%%%%%%%%%%%%%%%%%%%%%%%%%%%%%%%%%%%%%%%%%%%%%%%%%%%%%%%%%%%%%%%%%%%%

\section{معرفی گیت و عملیات اصلی آن}\ \\

\textbf{:Git} \newline
گیت یک سیستم مدیریت کد منبع توزیع شده است که میتوانید نوشته‌ها وکدهایتان را با آن در سیستم شخصی خودتان مدیریت کنید و تغییرات کدهایتان را داشته باشید، به تغییراتی در گذشته برگردید مثل  به ریلیز خاصی از پروژه، کدها را روی یک سرور گیت ( مانند گیت هاب) با دیگران سهیم شوید و گروهی روی توسعه‌ی یک پروژه همکاری کنید و از تغییراتی که هر عضو روی پروژه می‌دهد آگاهی یابید.به پروژه‌هایی که در آنها گیت استفاده میشود مخزن (repository) می‌گویند. بک مخزن گیت حاوی تمامی کدها، تغییرات کد و تنظیمات گیت برای آن پروژه است.در سیستم‌های توزیع‌شده هرکسی که به مخزن اصلی دسترسی دارد، می تواند یک کپی از مخزن اصلی را در اختیار داشته و تغییرات
خود را روی آن اعمال کند و همچنین می‌تواند این تغییرات را با تغییرات بقیه‌ی اعضا ترکیب کرده و یا به مخزن اصلی اضافه کند.  \newline

امروزه اشخاص و شرکت‌های کوچک و بزرگ زیادی از سیستم مدیریت کد منبع گیت برای کنترل کدهای نرم‌افزارها، پروژه ها و مستندات استفاده می‌کنند. کرنل لینوکس ، زبان برنامه‌نویسی روبی و فریمورک  راول نمونه ای از پروژه هایی هستند که بر روی سرویس گیت هاب نگهداری می شوند.گیت به تغییرات کدها تنها به عنوان چند خطی که تغییر میکنند نگاه نمی‌کند و پس از ثبت هر تغییر، کامیت (commit) یک تصویر کلی از پروژه را در لحظه‌ی آن تغییر ذخیره میکند و بررسی تغییرات با اطمینان و سرعت بالاتری امکان‌پذیر می‌شود.همنین گیت برخلاف بسیاری از سستم‌های مدیریت کد منبع وابستگی به سرور اصلی ندارد و تقریبا هرکاری را میتوان بصورت محلی (local) انجام داد، در هر لحظه ای و هر شرایطی کافیست تنها گیت را روی سیستم خود داشته باشید، تغییراتتان را ثبت کنید و هر زمان که به شبکه‌ای که سرور گیت شما در آن قرار دارد ( مانند اینترنت!) دسترسی داشتید می‌توانید تغییرات ذخیره شده را به مخزن روی سرور اضافه کنید و البته تمام تغییرات پروژه را هم بدون نیاز به اینترنت در مخزن محلی خود داشته باشید.استفاده از گیت بسیار ساده است. تیم توسعه‌ی گیت یک نرم‌افزار تحت خط فرمان برای استفاده از گیت ساخته است. همچنین پروژه‌های زیادی هم برای کار با گیت چه به‌عنوان افزونه (plugin) برای ویرایشگرها و محیط‌ها مجتمع توسعه و چه بصورت برنامه‌های جدا با رابط گرافیکی توسعه داده میشوند و امروز تقریبا در هرمحیطی می‌توان روشی برای استفاده از گیت یافت. \newline

دلیل استفاده از گیت چه میباشد ؟\newline
احتمالا  تجربه از دست دادن بخشی از پروژه را به دلایل گوناگون مثل ذخیره ناخواسته در هنگام قطعی برق یا تغییراتی که بعدا باعث بروز مشکل شده‌اند و دلایل بیشمار دیگر را داشته‌اید. به‌عنوان راه‌حل هم شاید روش‌هایی مانند کپی‌گرفتن از کل فایل‌ها واطلاعات پروژه دزمان‌های مختلف برای حفظ حالت خاصی از تغییرات را استفاده کرده باشید.این کار تا حدی جواب میدهد، اما در این صورت با انبوهی از دایرکتوریهایی که مانند یک غول بیشاخ و دم بزرگ میشوند چه میکنید؟ و از آن بدتر چگونه آن را با افراد دیگری که با شما در انجام آن همکاری میکنند مشترک میشوید؟ابزارهای مدیریت کد منبع پاسخی برای این شلختگیها و شلوغیهاست. با استفاده از این ابزارها میتوانید هرلحظه ای که مایل بودید تغییرات خود را ثبت کنید، به تغییرات ثبت شده در گذشته برگردید و به راحتی با دوستان و افراد تیمتان روی پروژه ای همکاری کنید، بدون اینکه نگران بهمریختگی و نامنظم شدن کدهای پروژه باشید. میتوانید برای هر تغییر که ثبت میکنید توضیحاتی بنویسید، تغییراتی که دیگران در پروژه لحاظ نموده اند را ببینید و البته منشاء باگها و خطاهای احتمالی را به سادگی یافته و رفع و رجوع کنید.در این میان ابزارهای زیادی برای این‌دست ‌کارها ساخته شده است که هرکدام نگاه و شیوه‌ی متفاتی را برای حل این مشکل در پیش گرفته است. معروفترین این ابزارها عبارتند ازmercurial ،svn ، gitو cvs که البته در این میان git یکی از جوانترین و پرطرفدارترین ابزارهای مدیریت کدمنبع
است و ویژگیهای ساده و متمایز آن باعث شده عمده‌ی تیم‌های نرم‌افزاری در دنیا به استفاده از این ابزار خوب روی بیاورند. \newline

برخی از امکانات گیت عبارت است از؛ \newline
خط فرمان : برنامه کامپیوتری که برای وارد کردن دستورات گیت استفاده میکنیم. توی مک Terminal نامیده میشه، روی پی سی برنامه‌های غیر بومی هست که هنگامی که برای اولین بار گیت رو دانلود میکنید، اون رو هم دانلود خواهید کرد. و روی سیستمهای لینوکسی از ترمینال استفاده خواهیم کرد. که در هر حالتی ما دستورات متنی را به جای استفاده از موس روی صفحه تایپ خواهیم
کرد. \newline
مخزن : پوشه یا فضای ذخیرهسازی که پروژه شما داخل اون وجود دارد. بعضی وقتها کاربران گیتهاب از اون با عنوان repoنام میبرند. اون میتونه یه پوشه روی کامپیوتر شما باشه یا فضایی روی گیتهاب یا هر سرویس میزبانی آنلاین دیگری. شما میتونید فایلهای برنامه نویسی، متنی، عکس و هرچیزی رو داخل مخزن ذخیره کنید. \newline
کنترل نسخه :هدف اساسی که گیت برای اون طراحی شد. وقتی یه فایل ورد مایکروسافتی دارید، مجبورید که هر بار که مجزا ذخیره‌اش کنید یا چندین نسخه از اون رو دخیره کنید. با گیت، مجبور به این کار نیستید. اون اسنپشات هایی از هر لحظه از تاریخچه‌ی پروژه رو ذخیره میکنه، بنابراین هیچ وقت هیچ چیزی رو از دست نمیدین با بازنویسی نمیکنید. \newline
سپردن : کامیت رکوردی است از فایلهایی که از زمان کامیت قبلی تا به اینجا تغییر داده‌اید. در اصل شما تغییرات را در مخزن (repo) خود اعمال میکنید و به git میگویید که این تغییرات را در یک کامیت ذخیره کند. کامیتها اساس پروژه شما را تشکیل میدهند و این اجازه را به شما میدهند که به هر وضعیتی از پروژه در هر نقطه از زمان باز گردید. وقتی که Commit میکنید، اسنپشاتی از وضعیت فعلی پروژهتون رو در نقطهی زمان فعلی ایجاد میکنید، که به شما نقطهی بررسی برای تجدید نظر یا بازگرداند پروژه به این نقطه رو خواهد داد. \newline
شاخه : فرض کنید میخواهید یک ویژگی جدید به برنامه اضافه کنید و نمی‌خواهید پروژه‌ی اصلی را تغییر دهید. اینجا جاییست که ساختن شاخه یا branch به کمکمان می‌آید. شاخه‌هابه شما اجازه می‌دهند که بین وضعیتهای پروژه جا‌به‌جا شوید. برای مثال اگر میخواهید یک صفحه‌ی جدید به وب سایتتان اضافه کنید می‌توانید بک شاخه‌ی جدید فقط برای آن صفحه ایجاد نمایید بدون این که بخش اصلی پروژه را تحت تاثیر قرار دهید. وقتی ساخت این صفحه به پایان رسید میتوانید تغییراتتان را از شاخه‌تان با شاخه اصلی پروژه ترکیب (merge) کنید. وقتی یک شاخه‌ی جدید میسازید، گیت کامیتی را که شاخه‌ی شما از آن جدا شده نگه میدارد و بنابراین تاریخچه‌ی تمامی
فایلها را دارد.