%%%%%%%%%%%%%%%%%%%%%%%%%%%%%%%%%%%%%%%%%%%%%%%%%%%%%%%%%%%%%%%%%%%%%
%%%%%%%%%%%                   PROBLEM 4                   %%%%%%%%%%%
%%%%%%%%%%%%%%%%%%%%%%%%%%%%%%%%%%%%%%%%%%%%%%%%%%%%%%%%%%%%%%%%%%%%%

\section{نحوه‌ی استفاده از فرسنگ نما و برچسب در یک پروژه‌ی نرم افزاری}\ \\

یک فرسنگ نما یک اتفاق خاص و یا دستاورد مهم و ارزشمند در طی انجام یک پروژه‌ی نرم افزاری است. فرسنگ‌نما‌ها ابزار‌ مناسبی برای دنبال کردن پیشرفت یک پروژه‌ی نرم افزاری، بررسی وضعیت پیشرفت آن و انتقال انرژی و انگیزه به افراد پروژه‌ی نرم افزاری برای ادامه‌ی کار هستند. برای تعینن فرسنگ‌نما‌ها در یک پروژه‌ی نرم افزاری باید ابتدا یک ساختار شکست کار
\LTRfootnote{Work Breakdown Structure (WBS)}
را ایجاد کرد و از طریق آن پروژی نرم‌افزاری را به بخش‌های کوچکتر و با قابلیت مدیریت بیشتری تقسیم کرد. سپس بایستی شروع به قرار دادن فرسنگ‌نما‌هایی در اطراف نقاط مهم و کلیدی در سیر پیشرفت پروژه کرد. نمایش تصویری پیشرفت در دستیابی به این فرسنگ‌نما‌ها می‌تواند عامل موثری در بهبود چینش‌ آنها و همچنین ایجاد انگیزه در تیم برای پیشبرد کار باشد.
 در استفاده از فرسنگ‌نما‌ها باید نکاتی را مورد توجه قرار داد :

\begin{enumerate}
\item 
بایستی در انتخاب رخداد‌های مهم در پروژه و task ها به عنوان فرسنگ نما دقت لازم را به خرج داد. نبایستی دستیافت‌های سطح پایین و کوچک را به عنوان فرسنگ نما تعیین کرد که این کار باعث می‌شود مفهوم فرسنگ‌نما ارزش تشویقی خود را از دست بدهد. همچنین نباید در انتخاب فرسنگ‌نما ها تفریط به خرج داد و اتفاقات مهم را به عنوان فرسنگ نما انتهخاب نکرد.
\item 
بایستی فاصله‌ی میان فرسنگ‌نما‌های متوالی را به درستی تنظیم کرد. همانطور که اشاره شد، فاصله‌ی کم بین فرسنگ‌نما‌ها باعث از دست رفتن معنا و مفهوم فرسنگ نما خواهد شد و فاصله‌ی بسیار زیاد بین‌آنها و نادیده گرفتن دستاورد‌های تیم نرم‌افزاری به عنوان فرسنگ نما باعث دلسردی آنها در ادامه خواهد شد. به طور متوسط مناسب است که برای پروژه‌های چند ماهه فرسنگ‌نما‌ها بیشتر از ۲ هفته با هم فاصله نداشته باشند.
\item
فرسنگ‌نما‌ها باید به صورت واضح و مشخص در روند انجام پروژه دیده شوند و به صورت مرتب بررسی گردند.
\item
اگر بنا به دلایلی فرسنگ نمایی در زمان مقرر به دست نیامد باید سریعا علل این موضوع بررسی شده، مجددا نقشه‌ای طرح شود و با تخمین دقیق تر منابع و اختصاص کافی آنها به آن پرداخته شود.

\end{enumerate}

برچسب‌گذاری برای فعالیت‌های تیم نرم‌افزاری یک فعالیت عام تر از تعیین فرسنگ‌نما است. به این صورت تعیین فرسنگ نما نوعی برچسب‌گذاری خاص تلقی می‌شود.

با استفاده از برچسب گذاری می‌توان کار‌ها را در یک تیم نرم‌افزاری دسته بندی‌کرد، مسئول انجام هر‌کار را تعیین نمود یا برچسب‌های زمانی و یا توضیحات مناسب در مورد  هر فعالیت و کار را به آن اضافه کرد.

امروزه استفاده از برچسب گذاری به صورت فیزیکی و مجازی در بورد‌های مدیریت تیم در تیم‌های نرم‌افزاری بسیار رایج است.

\subsection*{منابع} \ \\
\lr{\url{https://www.brighthubpm.com/project-planning/68427-successful-project-milestone}}

\lr{\url{-planning/}}

\lr{\url{https://www.smartsheet.com/blog/support-tip-milestones-in-project-management}}

\lr{\url{http://chambers.com.au/glossary/milestone.php}}


