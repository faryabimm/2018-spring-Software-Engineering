%%%%%%%%%%%%%%%%%%%%%%%%%%%%%%%%%%%%%%%%%%%%%%%%%%%%%%%%%%%%%%%%%%%%%
%%%%%%%%%%%                   PROBLEM 1                   %%%%%%%%%%%
%%%%%%%%%%%%%%%%%%%%%%%%%%%%%%%%%%%%%%%%%%%%%%%%%%%%%%%%%%%%%%%%%%%%%

\section{معرفی و مقایسه‌ی نمونه‌های مشابه سامانه‌ی مکان‌محور مدیریت کارمند}

\subsection{نمونه‌ی ۱:‌سامانه‌ی «کارمند کنترل»}

\subsubsection{آدرس}
\url{karmandcontrol.com}

\subsubsection{بیان هدف}
هدف از این سامانه برطرف سازی دغدغه های مدیران برای تسلط و نظارت بر بخش های مختلف کسب و کار می باشد. با این سامانه به جای استخدام نیروهای نظارتی با استفاده از راهکاری هوشمند و نرم افزاری می توان مدیریت بخش های منابع انسانی و حراست ومالی و … را انجام داد.

\subsubsection{نوع درآمد}
درآمد این سایت از طریق فروش خدمات و امکانات خود صورت می گیرد. میزان هزینه خدمات براساس مدت ارائه و حجم آن‌ (میزان افرادی از سازمان که درگیر سامانه می شوند) افزایش می یابد. منبع درآمد از ۳ خدمتی است که سامانه ارائه کرده است که شامل پنل مدیریت کارمدان, پنل مدیریت بازاریابان و پنل ترکیبی می باشد.

\subsubsection{تعداد افراد درگیر}
افراد درگیر در سامانه علاوه بر مدیران یک سیستم کاری, شامل کارمندان و بازاریابان و در کل پرسنل این سیستم کاری می باشد. زیرا این سیستم زمینه های مختلفی شامل حضور و غیاب, میزان فعالیت, مدیریت هزینه ها, محاسبه دستمزد, مکان یابی, مدیریت مشتری و سفارشات, تحلیل و آنالیز کلیه فعالیت های انجام شده و … علاوه بر مدیریت کارمدان و بازاریابان را برای سیستم کاری فراهم می کند. ضمنا علاوه بر افراد سیستم کاری, از طرف سامانه نیز افرادی به عنوان پشتیبانی از سامانه در آن درگیر خواهد شد.

\subsubsection{میزان موفقیت}
در مورد میزان موفقیت این سامانه همچنان نظر دقیقی نمیتوان داد. نسخه اولیه این وبسایت یک سال و ۶ ماه است که ارائه شده است. با این حال در این مدت با اینکه فعالیت لازم را در فضای مجازی داشته اند و هم اکنون کانال های تلگرام و اینستاگرام آنها فعال است ولی دنبال کننده زیادی ندارند. با اینکه این معیار مناسبی برای سنجش موفقیت این سیستم نمی باشد ولی فعالیت مناسب آن نشان دهنده این است که جذب مخاطب درستی از فضای مجازی نداشته است. با رجوع به رتبه الکسای این وبسایت می توان در یافت که فعالیت های ۶ ماه اخیر سایت بسیار افزایش داشته است و میزان ترافیک بر روی آن به میزان چشمگیری افزایش یافته است. 
البته بدون دنبال کردن شاخص های کمی مراجعه سامانه, باید گفت که محیط رابط کاربری سامانه بسیار مناسب طراحی شده است و تمامی بخش های آن بصورت کامل و با گرافیکی بالا طراحی شده اند. در کنار اینها بروزرسانی بلاگ ها و شبکه های مجازی سامانه نشان از پشتیبانی فعال دارد. لذا با همه اینها انتظار می رود که در ۶ ماه آينده این سامانه مورد توجه بیشتری قرار گیرد. البته قیمت نسبتا بالای خدمات نیز از عواملی است که شاید توجهات را از این سامانه می کاهد.

\subsubsection{سیستم عامل}
این نرم افزار قابل ارائه بر روی دستگاه های کامپیوتری و موبایلی می باشند. این سامانه تنها از سیستم عامل اندروید و ویندوز پشتیبانی می کند و میتوان آن را بروی کامپیوتر و موبایل و تبلت استفاده کرد. همچنین تحت وب نیز می توان تمامی امکانات را مدیریت کرد.

\subsubsection{زبان برنامه نویسی}
زبان Java و استفاده از \lr{Android Studio}
طراح \lr{UI/UX}
استفاده از WordPress و SEO
زبان برنامه نویسی سیستم تحت ویندوز این سامانه یافت نشد.

\subsubsection{متن باز بودن}

این سامانه, نرم افزاری متن بسته است که با توجه به هدف درآمدزایی آن از طریق فروش امکانات مورد انتظار بود.

در پایان باید این را بیان کرد که شرکت اصلی این سامانه شرکتی است که ارائه دهنده خدمات ردیابی برای دستگاه ها و اشخاص مختلف شامل ماشین, کودکان, سالخوردگان, حیوانات و … می باشد. با توجه به موفقیت اولیه این شرکت در امر ارائه خدمات ردیاب, این شرکت به ارائه سامانه ای به منظور کنترل کارمند رو آورده است که ردیابی بازاریابان تنها یکی از امکانات آن است. همچنین سامانه فوق افزونه های دیگری نیز دارد که با توجه به خواست مشتری می تواند به صورت اختصاصی ارائه شود.

\subsubsection{معرفی قابلیت‌های اصلی}

\begin{enumerate}
\item
 سیستم مدیریت مشتریان
\item
  گزارش حضور غیاب, میزان تاخیر و اضافه کاری
  \item
   مشاهده موقعیت لحظه ای بر روی نقشه و تیهیه گزارش روزانه از مسیرهای پیموده شده هر بازاریاب
   \item
    تحلیل عملکرد, ارائه بازده و ضریب عـملکرد مفید براى تک تک بازاریابان و کارمندان
    \item
     نـمایش تمامی گـزارش هــا در قـالب دیاگرام و نـمـودار جـهت تحلیل و بررسى سریع 
     \item
      قابـلیـت مشاهده صفحه نمایش سیستم کـارمنـدان و ارتبـاط با آنها 
      \item
       مدیـریت پـهنـاى بـانـد اینترنت و گزارش ترافیک مصرفى هر کارمند و بازاریاب
       \item
        سیستم تعریف وظیفه برای هر یک از بازاریابان، نظارت بر انجام آن و نمایش نتیجه کار برای مدیران
        \item
         گـزارش کـلیه ارتـباطـات مـتـنى و تلفنى به همراه تصویر محیط کارى کارمند و بازاریاب
         \item
          سیستم پیشرفته کـنتـرل قطعات سخت افزارى ویژه واحد کـنترل اموال


\end{enumerate}
\subsection{نمونه‌ی ۲:‌سامانه‌ی «چکینو»}

\subsubsection{آدرس}
 \url{checkino.ir}

\subsubsection{بیان هدف}
هدف از این سامانه حذف کاغذبازی و تجهیزاتی مانند کارت خوان و یا اسکنر انگشت و مدیریت حضور غیاب بصورت هوشمند می باشد. همچنین توسط این سامانه میتوان عملکرد و وظایف کارمندان را مدیریت کرد.

\subsubsection{نوع درآمد}
در آمد این سامانه از طریق فروش امکانات خود می باشد. در ابتدا امکان ۱۴ روز استفاده رایگان از نرم افزار را به کاربر می دهد و پس از آن کاربر می تواند در سه بسته مختلف با امکانات و قیمت های مختلف از این نرم افزار استفاده نماید.

\subsubsection{تعداد افراد درگیر}
افراد درگیر در سامانه علاوه بر فردی که ثبت نام کرده است و به عنوان کنترل کننده یا مدیر شرکت در نظر گرفته می شود, کارمندان شرکت کاری نیز درگیر سامانه هستند و نرم افزاری بر روی گوشی آنها نصبمی شود ککه وظیفه باید با آن تعامل داشته باشند. از دیگر افراد درگیر این سامانه تیم پشتیبانی مشتریان می باشند که اختلالات سیستم و درخواست های مشتریان را انجام می دهد. همچنین طبعا اگر از دیدگاه توسعه دهندگان سیستم نگاه کنیم, این افراد نیز درگیر گسترش فضای سیستم می باشند.

\subsubsection{میزان موفقیت}
در مورد موفقیت این نرم افزار باید آن را به نوعی ناموفق قلمداد کرد. پس از دو سال از راه اندازی آن رتبه آن در الکسا بسیار بالا بوده که نشان از این است که سامانه فوق موفق به جذب مشتری نشده است. عدم توضیحات کافی و اکتفا به توضیحات نرم افزار در صفحه اول, عدم ایجاد کاتالوگ برای معرفی نرم افزار, امکانات کم, وجود امکاناتی که نوشته شده اند ولی قابل بهره برداری نیستند شامل کنترل اطفال و سالخوردگان و … از جمله دلایل ضعف این نرم افزار می باشد.

\subsubsection{سیستم عامل}
این نرم افزار قابل ارائه تحت وب و بر روی تلفن های همراه اندرویدی می باشد.

\subsubsection{زبان برنامه نویسی}
زبان Java و استفاده از Android Studio برای طراحی اندروید
زبان های وب برای طراحی  frontend و backend  قالب و نرم افزار تحت وب 

\subsubsection{متن باز بودن}
این سامانه, نرم افزاری متن بسته است که با توجه به هدف درآمدزایی آن از طریق فروش امکانات مورد انتظار بود.

در پایان باید این نکته را بیان کرد که این سامانه از نظر جذب مشتری بسیار ضعیف عمل کرده است. بازاریابی ضعیفی داشته و به نظر پشتیبانی ضعیفی نیز دارد. امکانات کم نرم افزار باعث عدم تمایل مشتریان به استفاده از آن خواهد شد.


\subsubsection{معرفی قابلیت‌های اصلی}

\begin{enumerate}
\item 
 نظارت مستمر بر بازاریابان
 \item
  مدیریت کارمندان دورکار
  \item
   ردیابی مسیر رفت و برگشت سرویس مدرسه
   \item
    حضور  و غیاب خودکار
    \item
     امکان تعریف شیفت کاری و ماموریت های مختلف
     \item
      گزارش های مختلف شامل کارکرد کارمندان و ردیابی و پروژه ها و ورود و خروج
      \item
       امکان مدیریت تقویم و تعطیلات
       \item
        امکان ایجاد درخواست مرخصی
        \item
         ارائه گزارش پروژه ها و مدیریت آنها
         \item
          امکان تعریف شیفت شناور
\end{enumerate}


\subsection{مقایسه‌ی دو سامانه}
هر دو سامانه قابلیت تعریف کارمند و کنترل حضور و غیاب وی را دارد. همچنین هر دو می توانند با اپلیکیشنی که بر روی گوشی کارمند متصل است مکان وی را بسنجند. اما امکانات سامانه کارمند کنترل بسیار گسترده از امکانات چکینو است. چکینو به نوعی نیازمندیهای اولیه یک شرکت را برآورده می کند و می تواند نهایتا مناسب شرکتی کوچک با کارمندانی محدود باشد. اما سامانه کارمند کنترل, کنترلی دقیقی بر کلیه رفتارهای سیستمی کارمندان خود دارد. این سامانه کارمندان را از بازاریابان مستقل فرض میکند و امکانات مختلفی اعم از کنترل تماس و پیامک و تلفن و همچنین میزان ترافیک مصرفی با جزئیات را می تواند کنترل کند. همچنین این سامانه مدیریت هزینه ها و حقوق را برای کارمندان و شرکت محاسبه می کند.
بصورت کلی اگر بخواهیم این دو سامانه را مقایسه کنیم, سامانه چکینو یک نمونه کوچکی از سامانه کارمند کنترل است که البته طبعا هزینه استفاده از آن نیز کمتر است.

