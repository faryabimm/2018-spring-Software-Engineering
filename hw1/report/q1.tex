%%%%%%%%%%%%%%%%%%%%%%%%%%%%%%%%%%%%%%%%%%%%%%%%%%%%%%%%%%%%%%%%%%%%%
%%%%%%%%%%%                   PROBLEM 1                   %%%%%%%%%%%
%%%%%%%%%%%%%%%%%%%%%%%%%%%%%%%%%%%%%%%%%%%%%%%%%%%%%%%%%%%%%%%%%%%%%

\section{پرسش اول}\ \\
متودولوژی UP یک متودولوژی سنگین وزن شی گراست که از ۴ فاز اصلی تشکیل شده:
\begin{enumerate}
	\item Inception
	\item Elaboration
	\item Construction
	\item Transition
\end{enumerate}
در هر یک از این فاز‌ها اقدامات توسعه‌ی سیستم به صورت iterative انجام می‌شوند.
\begin{itemize}
\item 
در مرحله‌ی Inception سیستم هنوز در حال شناسایی‌است و نیاز‌مندی‌ها در حال استخراج هستند. در این حال هنوز مدلسازی‌ها دقیق نشده اند و وظیفه‌مندی‌های سیستم پیاده سازی نشده اند. تغییر در نیازمندی‌ها در این فاز کمترین هزینه را دارد و بسیار هم رایج و محتمل است.
\item
در مرحله‌ی Elaboration شناخت اولیه‌ی سیستم انجام شده و یک سری نیازمندی خیلی مهم و حیاتی سیستم پیاده سازی می‌شوند و مدل‌های سیستم دقیق می‌شوند. نیازمندی‌های حیاتی پیاده سازی شده احتمال تغییر بسیار کمی دارند و سایر نیازمندی‌ها هم در قالب مدل طراحی شده اند. تغییر در نیازمندی‌ها در این فاز معادل تغییراتی در مدل‌هاست که بسیار کم هزینه تر از تغییر در کد سیستم نرم افزاری است.
\item
در مرحله‌ی Construction سیستم در حال شکل گیری فیزیکی است و نیازمندی‌ها به صورت شدید در حال پیاده سازی هستند. تغییر در نیازمندی‌ها در این فاز ممکن است باعث تغییرات عمیقی در پیاده سازی و مدل‌های تهیه شده شود و از این رو هزینه‌ی زیادی به همراه خواهد داشت.
\item
در نهایت مرحله‌ی Transition مرحله‌ای است که در آن سیستم تهیه شده به شدت تست شده و به محیط کارفرما منتقل می‌شود. در این مرحله سیستم تقریبا به فرم نهایی خود دست یافته  و اعمال تغییرات در آن نسبت به سایر مراحل مستلزم پرداخت هزینه‌ی بسیار سنگین تری است!
\end{itemize}

در نهایت هر چه در فاز‌های متودولوژی UP به پیش برویم هزینه‌ی تغییر بیشتر و بیشتر خواهد شد.