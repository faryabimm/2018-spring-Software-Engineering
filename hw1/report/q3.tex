%%%%%%%%%%%%%%%%%%%%%%%%%%%%%%%%%%%%%%%%%%%%%%%%%%%%%%%%%%%%%%%%%%%%%
%%%%%%%%%%%                   PROBLEM 3                   %%%%%%%%%%%
%%%%%%%%%%%%%%%%%%%%%%%%%%%%%%%%%%%%%%%%%%%%%%%%%%%%%%%%%%%%%%%%%%%%%

\section{پرسش سوم}\
\subsection{}
توسعه‌ی component-based بسیاری از صفات مدل مارپیچی
\LTRfootnote{Spiral Model}
را به ارث برده است. این مدل ذاتا مدلی Evolutionary است. به بیان دیگر در این مدل تصریح می‌شود که رویه‌ی  تولید نرم افزار باید رویه‌ای iterative و تکاملی باشد. با این تفاوت که در این مدل به جای برنامه ریزی و مدل سازی و سپس تولید نرم افزار از کد پایه، نرم افزار از قطعات آماده‌ی متن باز یا خریداری شده ساخته می‌شود
\LTRfootnote{Commercial Of The Shelf Components}
.

بر خلاف روش‌های مارپیچی و تکاملی، در این روش مدلسازی با شناسایی قطعات احتمالی مورد نیاز در محصول شروع می‌شود. مدل Component-based  فارغ از روش و زبان تولید نرم افزار و متفاوت با روش‌های مارپیچی و تکاملی، مراحل زیر را برای تولید نرم افزار متصور است:
‌\begin{enumerate}
\item
محصولات آماده‌ی component-based برای استفاده در حوزه‌ی کاری مورد نظر مورد تحلیل و بررسی قرار می‌گیرند.
\item
مشکلات مربوط به ادغام بخش‌های مختلف نرم افزار سنجیده و در نظر گرفته می‌شود.
\item
معماری نرم‌افزاری مناسبی برای نگهداری واتصال قطعات طراحی و ساخته می‌شود.
\item
قطعات درون این معماری پایه‌ای قرار می‌گیرند و در نهایت آزمون‌های جامعی برای سنجش صحت و درستی عملکرد نرم‌افزار طراحی و اجرا می‌شوند.
\cite{pressman}

\subsection{}
مدل توسعه‌ی component-based باعث استفاده‌ی مجدد از کد و قطعات نوشته‌شده می‌شود. از این رو در صورتی که به عنوان بخشی از شالوده‌ی اصلی رفتار سازمانی در تولید نرم‌افزار جا بیفتد، می‌تواند در کاهش خطا‌های توسعه‌ی نرم‌افزار، کاهش هزینه‌ها و منابع تولید نرم‌افزار و همچنین زمان توسعه‌ی نرم افزار نقش به سزایی بازی کند!
\cite{pressman}
\end{enumerate}
