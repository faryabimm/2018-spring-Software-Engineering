%%%%%%%%%%%%%%%%%%%%%%%%%%%%%%%%%%%%%%%%%%%%%%%%%%%%%%%%%%%%%%%%%%%%%
%%%%%%%%%%%                   PROBLEM 2                   %%%%%%%%%%%
%%%%%%%%%%%%%%%%%%%%%%%%%%%%%%%%%%%%%%%%%%%%%%%%%%%%%%%%%%%%%%%%%%%%%

\section{پرسش دوم}\ \\

\subsection{آشنایی و توضیحات}

سرویس‌های PaaS 
\LTRfootnote{Platform as a Service}
سرویس‌هایی ابری هستند که طی آن‌ها ارائه‌ دهنده‌ی سرویس و خدمات، تمامی پلتفرم توسعه‌ی نرم‌افزار را در اختیار مشتری قرار می‌دهد. به این ترتیب مشتری می‌تواند به توسعه‌، اجرا و استفاده از نرم‌افزار‌های جدید یا موجود بپردازد، بدون آنکه نگران ساخت، نگهداری و مراقبت از زیر‌ساخت‌های سخت افزاری و نرم‌افزاری مورد نیاز باشد. به این ترتیب یک ارائه‌دهنده‌ی خبره‌ی خدمات این وظایف را بر عهده خواهد گرفت.

PaaS می‌تواند به ۳ طریق در اختیار مشتری قرار بگیرد:
\begin{enumerate}
\item
 یک سرویس ابری که توسط یک سرویس‌دهنده‌ی عمومی مانند Amazon در اختیار مشتری‌ها قرار می‌گیرد.
 
 در این نوع سرویس‌ها سرویس‌دهنده تمامی امکانات سخت افزاری و نرم‌افزار‌های اصلی را در قالب سرویس خود ارائه می‌کند. این نرم افزار‌ها شامل سیستم‌های عامل، سیستم‌های مدیریت پایگاه داده
 \LTRfootnote{DataBase Management Systems (DBMS)}
 سیستم‌های توزیع و مدیریت وظایف، ابزار‌های تولید و توسعه‌ی نرم افزار مثل JVM و انواع کامپایلر‌ها و ابزار‌های شبکه هستند.
 
\item
یک سرویس ابری خصوصی درون سازمانی.
\item
نرم‌افزاری که بر روی یک زیرساخت عمومی قرار داده می‌شود تا سرویسی را در اختیار جامعه‌ی هدف خود قرار دهد.
\cite{PaaS-Wikipedia}

\end{enumerate}
به عنوان مثال می‌توان این سرویس‌دهنده‌ها را برای PaaS معرفی کرد:
\begin{enumerate}
\item \lr{Amazon Web Services}
\item \lr{Sales Force}
\item \lr{Microsoft Azure}
\item \lr{Long Jump}
\item \lr{IBM Smart Cloud}
\item \lr{Open Shift}
\item \lr{Cloud Foundry}
\item \lr{Google App Engine}
\item \lr{CloudBees}
\item \lr{Engine Yard}
\item 
\end{enumerate}

\subsection{معرفی دقیق تر دو سرویس و مقایسه‌ی آنها}
 در این بخش دو سرویس محبوب تر \lr{Microsoft Azure} و \lr{Amazon Web Services} را معرفی و با هم مقایسه می‌کنیم:

سرویس \lr{Azure} چندین گزینه‌ی ممکن برای انتخاب از طرف توسعه‌دهنده‌ها را در حوزه‌ی PaaS ارائه می‌کند. از این سرویس‌ها می‌توان به  این موارد اشاره کرد:
\begin{enumerate}
\item سرویس‌های میزبانی برنامه‌های ابری \lr{(App Services)}
\item سرویس‌های ابری \lr{(Cloud Services)}
\item \lr{Server Fabric}
\item سرویس‌های مبتنی بر محفظه‌ها \lr{(Container based Services)}
\item سرویس‌های رویه‌ای و function محور.
\item سرویس‌های پردازش Batch.
\item سرویس WebJobs
\end{enumerate}

در مقابل AWS هم گزینه‌های مشابهی را در اختیار مشتریان خودر قرار می‌دهد. از آنها می‌توان این موارد را بر شمرد:
\begin{enumerate}
\item \lr{Container Service}
\item \lr{Elastic BeanStalk}
\item \lr{Lambda Service}
\item \lr{Batch Service}
\end{enumerate}

سرویس‌های AWS در زمینه‌ی میزبانی برنامه‌های کاربردی گزینه‌های چندانی را در اختیار کاربران قرار نمی‌دهند. از این رو در این بخش سرویس Azure حرف‌های بیشتری برای گفتن دارد.

سرویس‌های ابری Hybrid  سرویس‌هایی هستند که در آنها تکنولوژی‌های ابری روز با پایگاه‌های داده و نرم‌افزار های Legacy قدیمی کار می‌کنند.  بسیاری از شرکت‌ها به لحاظ اقتصادی تعویض سیستم‌های Legacy را موجه نمی‌دانند و از این رو چنین سرویس‌هایی مساعد و ایده‌آل برای این شرکت‌ها هستند. سرویس Azure در استفاده و به کار گیری این سرویس‌های Legacy در کنار سرویس‌های نوین ابری بهتر از سرویس‌های AWS عمل می‌کند.

برای کابران و توسعه دهندگان تکنولوژی‌های مربوط به Microsoft مانند  
\lr{Windows Server} 
و 
\lr{SQL Server}
و
\lr{.NET}
سرویس Azure سرویس بسیار بهتری است. اگرچه سرویس ‌AWS هم پشتیبانی خوبی از این حوزه دارد.

از طرف مقابل برای کابران و توسعه‌ دهندگان متن باز سرویس AWS سرویس مناسب تری است. شرکت Microsoft در گذشته هم آنچنان اقبالی به دنیای متن باز نداشته است. در مقابل Amazon طرح‌های پیشنهادی مناسبی برای توسعه‌دهندگان متن باز ارائه می‌کند.

در نهایت بسیاری از ویژگی‌ها و خدمات Azure معادلی در خدمات AWS دارند اما برخی از خدمات Azure هم منحصر به فرد هستند!
از این خدمات می‌توان به این موارد اشاره کرد:
\begin{enumerate}
\item \lr{Azure Visual Studio Online}
\item \lr{Azure Site Recovery}
\item \lr{Azure Event Hubs}
\item \lr{Azure Scheduler}
\end{enumerate}

در نهایت انتخاب بین این دو کاملا به نیاز‌های کاربر بستگی خواهد داشت.
\cite{Azure-vs-AWS}