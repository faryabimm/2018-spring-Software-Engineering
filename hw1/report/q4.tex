%%%%%%%%%%%%%%%%%%%%%%%%%%%%%%%%%%%%%%%%%%%%%%%%%%%%%%%%%%%%%%%%%%%%%
%%%%%%%%%%%                   PROBLEM 4                   %%%%%%%%%%%
%%%%%%%%%%%%%%%%%%%%%%%%%%%%%%%%%%%%%%%%%%%%%%%%%%%%%%%%%%%%%%%%%%%%%

\section{پرسش چهارم}
\subsection{\lr{Release Stage}}
\subsubsection{مساله}
ارائه و توزیع نرم افزار آماده شده به صورت مناسب و موثر در میان کاربران و افراد هدف.
\subsubsection{شرایط اولیه}
شرایط زیر برای استفاده از این الگوی فرایندی باید آماده باشد:
\begin{enumerate}
\item 
برنامه‌ی نرم‌افزاری باید برای ارائه‌ شدن آماده و package شده باشد.
\item
فرایند‌های ارائه و توزیع نرم افزار باید به درستی و با دقت تعریف شده باشند.
\item
برنامه‌ی آموزش کارمندان و کاربران هدف باید از قبل به صورت مدون آماده و پیشبینی شده باشد.
\item
مستندات کار با سیستم و مستندات پشتیبانی از آن باید در زمان توسعه‌ی سیستم نرم افزاری آماده شده باشند.
\end{enumerate}
\subsubsection{راه حل موثر برای ارائه‌ی سیستم نرم افزاری}
برای ارائه‌ی سیستم نرم‌افزاری باید اقداماتی انجام شود که هر یک شامل گام‌هایی هستند. این اقدامات می‌توانند بسته به ماهیت سیستم در دست توسعه بسیار ساده و پیش پا افتاده و یا بسیار پیچیده باشند.
\begin{enumerate}
\item
آماده شدن برای ارائه‌ی نرم افزار: 
\newline
در این بخش باید اقدامات لازم برای شروع فرایند ارايه انجام شود. ابتدا باید مطمئن شویم که همه چیز برای ارائه‌ی نرم افزار آماده و مرتب است و سپس باید اجازه‌‌های لازم برای ارائه‌ی نرم افزار را کسب کنیم.

این گام‌ها را طی خواهیم‌کرد:
\begin{itemize}
\item تایید برنامه‌های ارائه‌ی محصول و آموزش آن.
\item تایید مستندات کار با سیستم و پشتیبانی از آن.
\item بسته و پکیج بندی سیستم نرم‌افزاری
\item اتمام تبدیل داده‌های قدیمی برای استفاده در سیستم جدید
\end{itemize}
\item
ارائه‌ی نرم افزار به تیم پشتیبانی و عملیات:
\newline
پس از آماده سازی نرم افزار و کسب اجازه‌های لازم باید برای ارائه‌ی آن آماده شویم به این منظور ابتدا تیمی  از درون سازمان باید روی نرم افزار کار کنند تا مشکلات احتمالی نهایی آن قبل از رسیدن آن به دست کاربران برطرف شود. در شرایط خاص نرم افزار‌های 24/7 امروزی ممکن است ارائه‌ی نرم افزار به تیم پشتیبانی هم زمان با ارائه‌ی عمومی نرم افزار صورت پذیرد.

این گام‌ها را طی خواهیم‌کرد:
\begin{itemize}
\item آموزش اعضای تیم عملیات
\item اعمال و تدوین فرایند‌های تیم عملیات
\item آموزش اعضای تیم پشتیبانی
\item اعمال و تدوین فرایند‌های تیم پشتیبانی
\end{itemize}

\item
ارائه‌ی نرم افزار به جامعه‌ی کاربری:
\newline
این گام‌ها را طی خواهیم‌کرد:
\begin{itemize}
\item اعلام انتشار رسمی سیستم نرم‌افزاری
\item آموزش کاربران سیستم نرم افزاری برای کار کردن با آن
\item ارائه و اعمال نهایی افزار
\end{itemize}

\end{enumerate}

\subsubsection{وظایف و task های مدیریتی پروژه در این فاز}
\begin{enumerate}
\item مدیریت مرحله‌ی انتشار سیستم:
\newline
در این مرحله مدیریت پروژه باید برنامه‌ی کلاس‌های آموزش سیستم نرم افزاری را تدوین و آماده کند (بهتر است این برنامه از قبل آماده شده باشد) و طوری برنامه ریزی را انجام دهد که همه‌ی گروه‌های مخاطب برنامه از کابران هدف گرفته تا کارمندان با سیستم آشنا شوند.
\item مدیریت افراد:
\newline
یکی از مهمترین اقدامات مدیریتی در این بخش است. ممکن است انواع مختلفی از برخورد با سیستم نرم افزاری را از افراد مختلف جامعه‌ی هدف شاهد باشیم. از مقاوت در برابر گسترش سیستم و مقاومت در برابر استفاده از آن گرفته تا تلاش در استفاده‌ی نا درست از سیستم. بایستی با تدابیر مدیریتی درست این اقدامات مدیریت شوند.
\item آموزش و تمرین:
\newline
علاوه بر آموزش‌های در جهت استفاده از سیستم نرم افزاری، برخی اعضای تیم بایستی در زمینه‌های مدیریت افراد و ارتباط هم آموزش داده شوند. در مورد نحوه‌ی تعامل با مشتریان بالقوه و برقراری ارتباط صوتی، تلفنی یا رودررو با آنها گرفته تا تعامل با افرادی که در مقابل گسترش سیستم مقاومت می‌کنند.
\item کنترل کیفیت:
\newline
بایستی کیفیت تلاش‌های انجام شده در جهت آموزش افراد و کاربران محک زده شود و تاثیر آنها بررسی شود و در صورت لزوم مجددا جلسات آموزش و تمرین برگزار شود.

همچنین باید فرایند ارائه‌ی نرم افزار با زیر نظر گرفتن آزمایشی آن در برخی از نقاط مورد بررسی کیفی قرار گیرد و در صورت لزوم اصلاحات لازم در آن انجام شود.

\end{enumerate}

\subsubsection{ریسک‌ها و خطرات ممکن}
\begin{enumerate}
\item نادیده گرفتن یا کم رنگ دیدن تاثیر آموزش و تمرین در اثر کمبود وقت و منابع.
\item قبول مستندات ضعیف و کم کیفیت در اثر کمبود وقت.
\item تبدیل مدل‌های داده‌ای به صورت دیر هنگام یا نا موفق در اثر کمبود وقت و عدم اختصاص وقت کافی به آنها.
\end{enumerate}

\subsubsection{معیار‌های ارزیابی فرایند ارائه‌ی نرم‌افزار}
\begin{enumerate}
\item بهبود‌های انجام شده در هر ارائه‌ی نرم افزار
\item مشکلات رفع شده در هر ارائه‌ی نرم افزار
\item درصد کاربران آموزش و تمرین داده‌شده
\item متوسط زمان آموزش و تمرین به ازای هر نفر
\end{enumerate}

\subsubsection{شرایط نهایی}
این شرایط باید برآورده شده باشند تا یک توسعه‌ی نرم‌افزاری کامل تلقی شود:
\begin{enumerate}
\item مشتریان باید تمرین داده شده باشند.
\item سیستم نرم‌افزاری باید ارائه شده باشد.
\item مستندات باید تایید و ارائه شده باشند.
\item محیط پشتیبانی نرم‌افزار ارايه شده باید در اختیار تیم پشتیبانی قرار گرفته باشد.
\end{enumerate}
\cite{more-process-patterns}
\subsection{\lr{Maintain and Support Phase}}
\subsubsection{مساله}
پشتیبانی مناسب از نرم افزار ارائه شده با بر طرف سازی ایرادات و ارائه‌ی پشتبانی لازم به کاربران سیستم نرم افزاری
\subsubsection{شرایط اولیه}
شرایط زیر برای استفاده از این الگوی فرایندی باید آماده باشد:
\begin{enumerate}
\item 
سیستم نرم افزاری باید با موفقیت تحویل کاربران شده باشد.
\item
نرم افزار باید در جامعه‌ی کاربری تحویل شده باشد، مستندات آن تایید و ارائه شده باشند و کاربران برای کار با نرم افزار آموزش و تمرین دیده باشند.
\end{enumerate}

\subsubsection{راه حل موثر و شیوه‌ی انجام کار در این مرحله}
در این مرحله، ارائه‌ی پشتیبانی لازم به تیم مشتری حدود ۸۰ تا ۹۰ درصد وقت تیم نرم‌افزاری را به خود اختصاص خواهد داد، شناسایی ایرادات و برطرف کردن‌آنها و بهبود کلی نرم افزار و همچنین مدیریت کار و تولید مستندات مربوط هر کدام ۵ تا ۱۰ درصد از وقت تیم نرم افزاری را خواهند گرفت.

مرحله‌ی پشتیبانی از سه اقدام پیاپی تشکیل شده است:
\begin{enumerate}
\item دریافت و بررسی درخواست پشتیبانی.
\item بررسی بیشتر درخواست و رسیدن به پاسخ و راه‌حلی برای آن.
\item برطرف کردن مشکل با استفاده از راه حل.
\end{enumerate}

\subsubsection{وظایف و task های مدیریتی پروژه در این فاز}
\begin{enumerate}
\item مدیریت مرحله‌ی نگهداری و پشتیبانی:
\newline
در این فاز از پروژه‌ی نرم افزاری مباحث مدیریتی مربوط به پشتیبانی سیستم غالب هستند. بخش دیگر مباحث مدیریتی مربوط به برنامه ریزی جلسات تیم مدیریت پیکره‌ی سیستم نرم افزاری \lr{Configuration Control Board (CCB)} است. این  جلسات به صورت منظم برگزار می‌شوند و در آنها مشکلات و بهبود‌های احتمالی سیستم مورد بررسی قرار می‌گیرند.
\item مدیریت افراد:
\newline
موقعیت‌های مدیریتی در این فاز عبارتند از:
\begin{enumerate}
\item مدیریت پشتیبانی
\item مدیریت عملیات
\item مدیریت نیرو‌های فنی پشتیبانی
\item مدیریت نیرو‌های فنی عملیات
\item مدیریت تیم CCB
\end{enumerate}
\end{enumerate}

\subsubsection{ریسک‌ها و خطرات ممکن}

\begin{enumerate}
\item تغییرات تعیین شده در مرحله‌ی نگهداری سیستم عملیاتی نشوند.
\item بدون استفاده از چهارچوب نظام مند فرایند نرم‌افزاری در سیستم تغییراتی ایجاد شود.
\item به علت اقدام دیر هنگام تیم نرم افزاری و نیاز کابران، جامعه‌ی کاربری خودشان نسخه‌های جدیدی از نرم‌افزار کاربری را توسعه‌دهند.
\item 
\end{enumerate}

\subsubsection{شرایط نهایی}
این شرایط باید برآورده شده باشند تا یک سیستم نرم‌افزاری خارج از مرحله‌ی پشتیبانی و نگهداری تلقی شود:
\begin{enumerate}
\item نرم‌افزار با نسخه‌ی جدیدی جایگزین شده باشد.
\item بنا به دلایل فنی یا تصمیم مدیریتی، سیستم نرم‌افزاری از توسعه و کارایی خارج شده باشد.
\end{enumerate}
\cite{more-process-patterns}


