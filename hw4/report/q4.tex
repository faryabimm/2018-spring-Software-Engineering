%%%%%%%%%%%%%%%%%%%%%%%%%%%%%%%%%%%%%%%%%%%%%%%%%%%%%%%%%%%%%%%%%%%%%
%%%%%%%%%%%                   PROBLEM 4                   %%%%%%%%%%%
%%%%%%%%%%%%%%%%%%%%%%%%%%%%%%%%%%%%%%%%%%%%%%%%%%%%%%%%%%%%%%%%%%%%%

\section{پرسش چهارم}
SCM 
برای موفقیت یک پروژه‌ی نرم‌افزاری در تمامی متودولوژی‌های توسعه‌ی نرم‌افزار نقش کلیدی بازی می‌کند.
اصول اصلی SCM یعنی
شناسایی، کنترل، مشاهده و گزارش 
ضامن یکپارچگی و کیفیت محصول در دست تولید هستند.

می‌توان SCM را متناسب با نیاز فرایند و متودولوژی تولید نرم‌افزار به شیوه‌های متفاوتی پیاده سازی کرد (مادامی که اصول آن پابرجا بمانند).

در متودولوژی‌های اجایل بایستی مکانیزم‌های اعمال SCM دستخوش تغییراتی شوند که آن را با سرعت تغییرات و گردش‌های متودولوژی هماهنگ کنند.

به این صورت باید تمرکز بر پیاده‌سازی SCM به نحوی باشد که از release های کوتاه‌تر و سریعتر پشتیبانی کند، با ساخت‌های پیوسته
\LTRfootnote{Continuous Builds}
هماهنگ باشد، از فضا‌های کاری اشتراکی بین افراد و بخش‌های مختلف پروژه و تیم پشتیبانی کند، با چند شاخه‌شدن و ادغام‌های مکرر کد در روش‌های اجایل هماهنگ باشد، از خودکارسازی و اتوماسیون بالایی برخوردار باشد تا بتواند با سرعت بالای متودولوژی همراه شود، نقش‌های CM بایستی توسط پرسنل deployment انجام شوند و معیار‌هایی قابل اندازه‌گیری CM برای محصولات، فعالیت‌ها و وظایف تعریف و مدیریت و گزارش‌گیری شوند.

\subsection{گام‌های تغییر SCM برای متودولوژی‌های agile}
\subsubsection{پشتیبانی از گام‌های کوتاه‌تر}
SCM
باید از گام‌ها و افزایش‌ها و گردش‌های کوتاه‌تر پشتیبانی کند. به این صورت فرایند SCM باید بتواند به سرعت به شناسایی و کنترل تغییرات و المانها در سیستم نرم‌افزاری بپردازد و گلوگاهی برای سرعت متودولوژی اجایل نباشد در عین حال باید توان مورد نیاز برای مدیریت تغییرات را کمینه کند.
\subsubsection{پشتیبانی از گام‌های کوتاه‌تر}
برای ارائه‌ی ساخت‌های پیاپی می‌توان از ابزار‌هایی که به همین منظور طراحی شده‌اند استفاده کرد و به وسیله‌ی این ابزار‌ها سرعت بالای تغییرات و ساخت‌ها را مدیریت کرد.

همچنین نیاز به درک درستی از فعالیت‌های فعلی ساخت و انطباق آنها با انواع مختلف و مورد نیاز ساخت‌های موجود و سطوح مختلف آنها محسوس است.

این فعالیت به SCM در شناسایی محل هدررفت منابع و توان کمک شایانی می‌کند.

\subsubsection{محیط‌های کاری، چند شاخه شدن و ادغام کد}
سرعت بالای توسعه در متودولوژی‌های اجایل و ابعاد کار نیاز به وجود محیط‌های جدای کار برای افراد  و گروه‌های مختلف و محیط‌های اشتراکی لازم بین افراد و گروه ها و همچنین مکانیزم‌هایی برای ادغام این محیط‌ها را محسوس می‌کند.

ابزار‌های SCM امروزه با امکان
\lr{CM-coop environments}
عرضه می‌شوند که به تیم‌ها این امکان را می‌دهند که ساخت محیط جدید و ادغام محیط‌ها را به سادگی و بدون نیاز به تحمل سربار نصب ابزار‌های مدیریتی به ازای هر محیط انجام دهند.

\subsubsection{خودکار‌سازی}
سرعت بالای متودولوژی‌های چابک نیاز به سرعت بالای تمامی فعالیت‌های جانبی انجام شده در کنار آنها را ایجاب می‌کند.
فعالیت‌های SCM هم از این قاعده مستثنی نیستند. از این رو این فعالیت‌ها باید حتی الامکان تماما خودکار انجام شوند و دستی نباشند.


\subsubsection{انقال مدیریت تغییر به مرحله‌ی برنامه‌ریزی برای گردش}
تغییرات اعمال شده در پروژه نباید تا قبل از iteration بعدی رسیدگی شوند. 
جلسات رسیدگی به تغییرات تبدیل به جلسات روزانه با سایر اعضای تیم می‌شود و مدیریت تغییرات به مرحله‌ی  برنامه‌ریزی برای گردش بعدی انتقال می‌یابد.

\subsubsection{نقش‌های SCM}
این نقش‌ها در یک تیم چابک بایستی توسط افرادی انجام شود که بیشتر با تغییرات نهایی یک سیستم در حال توسعه و سرویس در ارتباط هستند. این افراد پرسنل deployment و نگهداری از سیستم هستند. این افراد در جریان تغییرات کلی و migration های قبلی سیستم هستند و تغییرات بعدی نیز باید به دست آنها انجام و اعمال شود.

\subsubsection{تعیین و تغییر معیار‌های SCM برای تشخیص هدررفت‌ها}
برای تشخیص هدررفت‌ها در توان و منابع تیم توسعه‌ی چابک باید معیار‌های SCM متناسب با شرایط چابک جدید تغییر کند. در اینجا نمونه‌ای از این معیار‌های تغییریافته آورده شده است:
\begin{itemize}
\item 
مقایسه‌ی \lr{user story} یا requirment هایی که برای توسعه و ارایه برنامه‌ریزی و اولویت بندی شده بودند با آنچه واقعا توسعه یافته و ارایه شده.
\item 
شناسایی تغییرات اضافه‌ای که برنامه ریزی نشده بودند ولی انجام شدند.
\item
اندازه گیری زمان صرف شده برای بازیابی از ساخت‌هایی که با شکست مواجه شده‌اند.
\item
...
\end{itemize}


\cite{cms-agile}