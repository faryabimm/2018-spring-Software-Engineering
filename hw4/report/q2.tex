%%%%%%%%%%%%%%%%%%%%%%%%%%%%%%%%%%%%%%%%%%%%%%%%%%%%%%%%%%%%%%%%%%%%%
%%%%%%%%%%%                   PROBLEM 2                   %%%%%%%%%%%
%%%%%%%%%%%%%%%%%%%%%%%%%%%%%%%%%%%%%%%%%%%%%%%%%%%%%%%%%%%%%%%%%%%%%

\section{پرسش دوم}
هر یک از جفت‌های تست و ابزار پیشنهادی را بررسی کرده و در مورد امکانات و نحوه‌ی کارکرد و نحوه‌ی تست و فراهم کردن نیاز‌های موضوع توسط ابزار توضیح می‌دهیم:

\subsection{تست رابط کاربری وب، TestingWhiz}
ابزار آزمون خودکار رابط کاربری وب TestingWhiz امکان  صحت سنجی رابط کاربری و front-end نرم‌افزار‌های تحت وب ، بررسی درستی عملکرد آن در نمایش UI و ارائه‌ی UX مناسب در واسط‌های کاربری متنوع و browser های متفاوت را به ما می‌دهد.

با استفاده از این ابزار می‌توان دو صفحه‌ی وب را capture کرده و در ادامه صحفه‌ی مورد تست را با صفحه‌ی هدف مقایسه کرد. این قیاس به صورت پیکسل به پیکسل و با دقت و حساسیت قابل تنظیم انجام می‌شود.

مقایسه‌ی انجام شده می‌تواند بین دو صفحه، یک صفحه و یک تصویر یا دو تصویر انجام شود.


به وسیله‌ی این ابزار می‌توان قابلیت جابه‌جایی بین صفحات
\LTRfootnote{navigation}
، چینش صفحات
\LTRfootnote{screen layout}
و قالب
\LTRfootnote{theme}
بخش‌های مختلف وبسایت را مورد بررسی قرار داد و از درستی و هم فرمی آنها اطمینان حاصل کرد.

همچنین می‌توان با استفاده از مکانیزم‌های ارائه شده، یکسانی و هم فرمی 
طراحی و 
\lr{look \& feel}
المان‌های مورد استفاده و برندینگ را مورد بررسی قرار داد.


مزایای استفاده از این ابزار تست عبارتند از:
\begin{itemize}
\item
افزایش سرعت آزمون رابط کاربری وب با نتایجی قابل اطمینان و درک.
\item
داشتن عملکرد و رابط کاربری یکسان و یک فرم در محیط‌ها و بخش‌های مختلف.
\item
فرایند user-friendly آزمون رابط کاربری.
\item
میزان دقت و حساسیت قابل تنظیم در تست‌ها برای انجام تست‌هایی مطابق با نیاز‌ها و دقت مورد انتظار.
\item
افزایش کیفیت محصولات ارايه شده و تجربه‌ی کاربری.
\end{itemize}

\imagecsref{5}{1}{web-ui-and-functional-testing-how-it-works-img}{فرایند آزمون رابط کاربری وب با استفاده از ابزار TestingWhiz}{https://www.testing-whiz.com/web-ui-comparison-and-functional-test-automation}




\subsection{تست API ، Postman}
ابزار Postman
یک محیط توسعه‌ی API یا به عبارتی دیگر یک ADE است
\LTRfootnote{API Development Environment}
.

این ابزار امکانات مناسبی در کنار محیط توسعه‌ی API برای تست خودکار آن ارائه می‌دهد.

ابزار Postman مجموعه‌ی کوچکی از خدمات API ها را یک collection و مجموعه‌ای  از collection ها را یک folder می‌نامد.

همچنین این ابزار امکانی ارائه می‌کند که به وسیله‌ی آن می‌توان یک تست را به یک collection یا folder تخصیص داد و به این صورت با اجرای هر‌یک از API  های آن مجموعه، تست مذکور اجرا خواهد شد.
به این ترتیب امکان reuse برای تست‌های نوشته شده وجود خواهد داشت.

می‌توان با استفاده از ابزار newman تست‌های نوشته شده خودکار را در فرایند CI/CD وارد کرد.

\begin{itemize}
\item
تست‌هادر ابزار postman با زبان JavaScript نوشته می‌شوند.

\item
تست‌ها در محیطی sandbox و غیر از محیط اجرای اصلی API هااجرا می‌شوند و از این رو انجام تست‌ها هیچ‌گونه مشکلی برای محیط اصلی ایجاد نخواهد کرد.
\item ین ابزار امکانی ارائه می‌کند که به وسیله‌ی آن می‌توان یک تست را به یک collection یا folder تخصیص داد و به این صورت با اجرای هر‌یک از API  های آن مجموعه، تست مذکور اجرا خواهد شد.
\item 
تست‌ها بعد از هر فراخوانی API به صورت خودکار اجرا خواهند شد و نتایج آنها برای تحلیل و آمارگیری ثبت می‌شود.
\end{itemize}

\imagecsref{6}{1}{WS-randomFullTests2}{تصویری از پنل تست در ابزار postman}{https://www.getpostman.com/docs/v6/postman/scripts/test_scripts}


\subsection{تست مقاومت در برابر اشکال در محیط ابر، سرویس‌های \lr{Netflix Simian Army}}

\lr{Netflix Simian Army}
شامل سرویس‌های 
\lr{Chaos Monkey}
،
\lr{Janitor Monkey}
و 
\lr{Conformity Monkey}
است. در ادامه به توضیح هر یک می‌پردازیم:
\cite{simian-army}

\subsubsection{\lr{Chaos Monkey}}
در سیستم‌ها و محیط‌های ابری معمولا چندین مولفه و زیر سیستم در تعامل با یکدیگر مغول به کار هستند یا برای توزیع بار از یک مولفه چندین نمونه در حال فعالیتند.

در چنین سیستم‌هایی مولفه‌ها باید به نحوی طراحی شوند که در صورتی که یک مولفه از کار افتاد کل سیستم فلج نشود و باقی مولفه‌ها در حد امکان به فعالیت و خدمت رسانی ادامه دهند.

\lr{Chaos Monkey}
در ساعات کاری روز‌‌های غیر تعطیل به صورت تصادفی یکی از مولفه‌های سیستم را انتخاب کرده و آن را از کار می‌اندازد تا رفتار باقی مولفه‌ها در قبال آن مورد بررسی قرار بگیرد.

علت این زمان بندی‌هم آن است که پرسنل نگهداری در حال کار باشند تا در صورت بروز مشکل بتوانند آن را رفع کنند.
\cite{chaos-monkey}

\subsubsection{\lr{Janitor Monkey}}
در سیستم‌های ابری منابع تقریبا غیرمحدودی در اختیار نرم‌افزار‌ها قرار دارد از این رو به سادگی می‌توان در مدیریت این منابع سردرگم شد و نسخه‌های قدیمی نرم‌افزار در حال اجرا، داده‌های غیر کاربردی و نسخه‌های پشتیبانی که دیگر مورد نیاز نیستند را کماکان نگهداری کرد.
این فضای اشغال شده‌ی اضافی تنها باعث اعمال هزینه‌های بیشتر به تیم توسعه‌ی نرم‌افزار خواهد شد.

\lr{Janitor Monkey}
به صورت پیش‌فرض ساعت ۱۱ صبح روز‌های کاری در بین منابع سیستم شروع به گردش می‌کند و هر منبع را در مقابل تعدادی از شرایط می‌سنجد. اگر هر یک از شرایط این موضوع که این منبع دیگر مورد نیاز نیست را به دست بدهند، آن منبع را برای پاک کردن نشانه گذاری می‌کند و پیامی به مالک منبع (در قالب ایمیل) می‌فرستد. طی این پیام به مالک منبع تاکید می‌شود که منبع بنا به شرایط محقق شده، اضافی‌است و ظرف مدت مشخصی که به صورت پیش‌فرض ۳ روز کاری است پاک خواهد شد. 

مالک منبع می‌تواند از طریق یک درخواست REST  جلوی این پاک شدن را بگیرد یا آن را پیش از موعد انجام دهد.

در صورتی که
\lr{Janitor Monkey}
به منبعی برخورد کند که برای پاک شدن نشانه‌گذاری شده و زمان پاک شدن آن گذشته است، در صورتی که state آن منبع عوض نشده باشد آن را پاک می‌کند.

شرایط پاک شدن یک منبع و بازه‌های زمانی و زمان شروع به کار \lr{Janitor Monkey} مطابق نیاز‌های کسب و کاری قابل تغییر است.
\cite{janitor-monkey}

\subsubsection{\lr{Conformity Monkey}}
در محیط‌های ابری و ‌self-scaling ممکن است بنا به دلایلی مانند عدم دانش کافی انسانی یا فراموشی نمونه‌هایی از محصول شروع به فعالیت کنند که مطابق الگو‌های به روز طراحی نشده‌اند یا مشکلاتی دارند که با قواعد conformity که قابل تنظیم هستند، هم‌خوانی ندارند.

\lr{Conformity Monkey}
در روز‌های کاری، هر ساعت یک بار به دنبال چنین نمونه‌هایی می‌گردد و در صورتی که نمونه‌ای را پیدا کرد که در شرایط آن صدق می‌کند، اطلاعات شرایط و نمونه‌ی پیدا شده و سیستم آن نمونه را به مالک سیستم و منبع ارسال می‌کند. به این ترتیب فرد مسئول می‌تواند اقدامات لازم را برای حل موضوع در دست اقدام قرار دهد.

شرایط conformity منبع و زمان کار \lr{Conformity Monkey} مطابق نیاز‌های کسب و کاری قابل تغییر است.
\cite{conformity-monkey}

