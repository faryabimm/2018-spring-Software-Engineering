%%%%%%%%%%%%%%%%%%%%%%%%%%%%%%%%%%%%%%%%%%%%%%%%%%%%%%%%%%%%%%%%%%%%%
%%%%%%%%%%%                   PROBLEM 3                   %%%%%%%%%%%
%%%%%%%%%%%%%%%%%%%%%%%%%%%%%%%%%%%%%%%%%%%%%%%%%%%%%%%%%%%%%%%%%%%%%

\section{پرسش سوم}
ابزار‌های مدیریت پیکربندی معرفی شده را با یکدیگر مقایسه می‌کنیم:
\cite{open-source-cmss}

\begin{center}
\begin{tabular}{|c|c|c|c|}
\hline
 & Ansible & Puppet & Chef \\ \hline
 زبان
 &Python&C++, Clojure&Ruby Ernalng, \\ \hline
گواهی استفاده(license)
 &\lr{GPLv3+}&\lr{Apache}&\lr{Apache 2.0} \\ \hline
احراز هویت دو‌طرفه
 &بله&بله&بله \\ \hline
رمزگذاری
 &بله&بله&بله \\ \hline
تایید بدون تغییر
 &بله&بله&بله \\ \hline
بدون agent
 &بله&خیر&خیر \\ \hline
رابط کاربری گرافیکی
 &بله&بله&بله \\ \hline
زبان توصیف سیستم
 &Yaml&\lr{a custom language}&Ruby \\ \hline
تاریخ انتشار
 &۲۰۱۲&۲۰۰۵&۲۰۰۹ \\ \hline
آخرین به روزرسانی 
 &ژانویه ۲۰۱۸&مه ۲۰۱۷&سپتامبر ۲۰۱۷ \\ \hline
\end{tabular}
\end{center}

\begin{itemize}
\item 
احراز هویت دو طرفه به معنی آن است که server ، client خود را تایید هویت کند و به صورت برعکس client ، server خود را هویت سنجی و تایید کند.
\item
agent داشتن به معنی این است که سرویس برای ارائه‌ی خدمات خود به یک سری پردازه‌ی daemon تحت عنوان agent در سیستم وابسته است.
\item
تایید بدون تغییر به این معنی است که سیستم بتواند اینکه یک المان یا منبع  مطابق قوانین سیتم یا محیط رفتار می‌کند یا پیکربندی شده است را بدون تغییر دادن آن المان و صرفا با خواندن اطلاعات آن تشخیص دهد.
\item
هر‌سه‌ی این سرویس‌ها روی سیستم‌های عامل و پلتفرم‌های رایج امروزی تست شده و کارایی مناسبی دارند.
\end{itemize}

\subsection{توضیحات تکمیلی}
\subsubsection{Ansible}
سیستم مدیریت پیکربندی Ansible فعالیت‌های استقرار چند گره‌ای
\LTRfootnote{Multi-node deployment}
اجرای  تک کاره (adhoc) وظایف و مدیریت پیکربندی را با هم ترکیب کرده‌است.
گره‌ها را از طریق SSH مدیریت می‌کند و برای این کار نیاز دارد که روی تک‌تک آنها Python نصب شده باشد.

\subsubsection{Puppet}
برای مدیریت گره‌ها از فراخوانی REST و پارادایم کارفرما-کارگزار استفاده می‌کند.
از مفهومی به نام 
\lr{resource abstraction layer}
استفاده می‌کند و به آن این امکان را می‌دهد که پیکربندی سیستم را به زبانی سطح بالا انجام دهد.

\subsubsection{Chef}
می‌توان از Chef در حالت‌های کارفرما-کارگزار یا تنها (solo) استفاده کرد



