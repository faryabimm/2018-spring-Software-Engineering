%%%%%%%%%%%%%%%%%%%%%%%%%%%%%%%%%%%%%%%%%%%%%%%%%%%%%%%%%%%%%%%%%%%%%
%%%%%%%%%%%                   PROBLEM 4                   %%%%%%%%%%%
%%%%%%%%%%%%%%%%%%%%%%%%%%%%%%%%%%%%%%%%%%%%%%%%%%%%%%%%%%%%%%%%%%%%%

\section{پرسش چهارم}
در این پرسش یک روند نوعی برای انجام 
\lr{Spring Retrospective}
معرفی و مورد بحث واقع می‌شود:

\vspace*{3cm}

\lr{Spring Retrospective}
به فرایندی اطلاق می‌شود که طی آن تیم نرم‌افزاری در متودولوژی scrum روند تطبیق فرایند‌های خود با فلسفه‌ی scrum را مورد بررسی قرار می‌دهند و در مورد تغییرات مورد نیاز برای ادامه‌ی راه به طرح راهکار می‌پردازند.

این جلسه با حضور تیم توسعه‌دهندگان، Scrum-master به عنوان گرداننده‌ی جلسه و Product-owner برگزار می‌شود. سایر افراد نباید به این جلسه دعوت شوند.

این جلسه باید در مکانی مناسب (مثلا اتاق جلسات شرکت نرم‌افزاری)  و پس از هر sprint برگزار شود. مدت زمان جلسه برای sprint های ۱ ماهه حداکثر ۳ ساعت و برای sprint های  ۲ هفته‌ای حداکثر ۱ و نیم ساعت است.


برای شروع جلسه مدیر جلسه بر روی یک تخته‌ی سفید روند زمانی sprint و وقایع مهم و milestone های آن را ترسیم می‌کند. همچنین باقی تخته به ۴ بخش تقسیم می‌شود که عبارتند از:

\begin{enumerate}
\item کار‌هایی که باید شروع به انجام آنها بکنیم
\item کار‌هایی که باید انجام آنها را ادامه دهیم
\item کار‌هایی که باید انجام آنها را متوقف کنیم
\item تشویق‌های مخصوص افرادی که تاثیر به سزایی در sprint گذشته داشته‌اند
\end{enumerate}


سپس بین حضار جلسه و به ازای هر نفر تعدادی sticky-note توزیع می‌شود و برای پر کردن این note ها بین ۵ تا ۱۰ دقیقه زمان تخصیص داده می‌شود. هر فرد باید در هر برگه تنها یک مورد از یکی از ۴ دسته‌ی مطرح شده را بنویسد. 

پس از پایان وقت افراد به نوبت پای تخته می‌آیند تا نکات مورد نظرشان را در دسته‌ی مورد نظر روی تخته قرار دهند و در مورد هر کدام توضیحی به جمع بدهند. پس از اتمام توضیحات آخرین نفر، نکات تکراری حذف می‌شوند و بر روی نکات ثبت شده در گروه‌ها به بحث و تبادل نظر پرداخته می‌شود و تصمیمات نهایی در گوشه‌ی دیگری از تخته ثبت می‌شوند.

خوب است در صورتی که در اولین جلسه‌ی Retrospective حضور داریم چکیده‌ای از محتوای جلسه‌ی قبل را هم در دسترس داشته باشیم تا انجام تصمیمات آن جلسه را مورد بررسی قرار دهیم. همینطور خوب است خلاصه‌ای از گزارش کار‌های انجام شده و پیش رو در انتها‌ی sprint در دسترس شرکت کنندگان در جلسه قرار گیرد.

در نهایت جمع بندی جلسه انجام می‌شود و تصمیمات گرفته شده برای اعمال به اعضای تیم ابلاغ می‌شود و در نهایت جلسه پایان می‌یابد.
\cite{retrospective}
