%%%%%%%%%%%%%%%%%%%%%%%%%%%%%%%%%%%%%%%%%%%%%%%%%%%%%%%%%%%%%%%%%%%%%
%%%%%%%%%%%                   PROBLEM 2                   %%%%%%%%%%%
%%%%%%%%%%%%%%%%%%%%%%%%%%%%%%%%%%%%%%%%%%%%%%%%%%%%%%%%%%%%%%%%%%%%%

\section{توضیح در مورد ارتباط المان‌های طراحی و موارد کاربرد}

در این بخش ارتباط هر یک از المان‌های طراحی به موارد کاربردی یا برعکس را بررسی خواهیم کرد. از آنجایی که در توضیح صفحات به صورت کامل به توصیف ارتباط موارد کاربرد پرداخته شد در اینجا از توضیح مواردواضح و پرگویی پرهیز خواهیم  کرد.

\begin{itemize}
\item \textbf{ثبت نام کاربر} \newline
این مورد کاربرد با استفاده از صفحه‌ی ثبت نام در سامانه پاسخگویی شده است.
به این صورت تنها این صفحه برای پاسخگویی به این مورد کاربردی طراحی شده و صفحه‌ی مذکور هم تنها همین کارایی را دارد.
\item \textbf{بازیابی کلمه‌ی عبور} \newline
برای بازیابی کلمه‌ی عبور فرایندی سه مرحله‌ای از ثبت درخواست، دریافت کلمه‌ی عبور و تعیین کلمه‌ی عبور جدید تعریف شده است. به این منظور سه صفحه برای آن طراحی شده که در بخش قبل مفصلا در مورد آن توضیح داده شد.
\item \textbf{ورود به سیستم} \newline
این مورد کاربرد با استفاده از صفحه‌ی ورود به سیستم در سامانه پاسخگویی شده است.
به این صورت تنها این صفحه برای پاسخگویی به این مورد کاربردی طراحی شده و صفحه‌ی مذکور هم تنها همین کارایی را دارد.
\item \textbf{خروج از سیستم} \newline
برای خروج از سیستم کافی است در هر لحظه و صفحه‌ای که در سامانه قرار داریم، گزینه‌ی خروج را از سمت راست نوار بالای صفحه انتخاب کنیم. به این صورت نشست کاربری به پایان رسیده و به صفحه‌ی اصلی منتقل خواهیم شد.
\item \textbf{ویرایش اطلاعات کاربری و خرید اشتراک سامانه} \newline
برای پاسخگویی به این دو نیاز از یک صفحه (حساب کاربری) استفاده کرده‌ایم. از آنجایی که خرید اشتراک سامانه به نوعی ارتقای حساب کاربری و ویرایش آن تلقی می‌شود و این خرید هم تنها یک بار خواهد بود یک جا کردن این دو مورد کاربرد را مناسب دیدیم.
\item \textbf{دعوت کارمند جدید} \newline
برای دعوت کارمند جدید کافی است در صفحه‌ی مدیریت کارمندان، دکمه‌ی عملیات قرمز رنگ پایین صفحه را انتخاب کرده و سپس در صفحه‌ی دعوت کارمند جدید یک مشخصه‌ی شناسایی او را وارد کنیم.
به این ترتیب درخواست دعوت برای وی ارسال خواهد شد.
\item \textbf{تخصیص و ویرایش شیفت کاری خاص} \newline
به این منظور کافی است از صفحه‌ی کارمندان کارمند مورد نظر انتخاب شود و در صفحه‌ی اطلاعات کارمند، گزینه‌ی  ویرایش شیفت کاری‌(نماد تقویم) انتخاب شود. به این ترتیب در صفحه‌ی ویرایش شیفت کاری کافی است روز‌های شیفت کاری و ساعات مورد نظر را از طریق رابط تقویم و ساعت ارائه شده انتخاب کنیم (به علت محدودیت سایت معرفی شده و پر شدن ظرفیت هر دو حساب کاربری مان نتوانستیم مدل اینصفحه را ارائه کنیم)
\item \textbf{ویرایش اطلاعات کارمند، شامل حذف} \newline
به این منظور کافی است از صفحه‌ی کارمندان کارمند مورد نظر انتخاب شود و در صفحه‌ی اطلاعات کارمند، گزینه‌ی  ویرایش اطلاعات کارمند (نماد کارت ویزیت) انتخاب شود. به این ترتیب در صفحه‌ی ویرایش کارمند کاری کافی است اطلاعات مورد نظر را وارد کنیم. همچنین برای حذف کارمند کافی است از طریق صفحه‌ی اطلاعات کارمند گزینه‌ی حذف را انتخاب کنیم.
\item \textbf{ثبت، ویرایش و حذف ردیاب} \newline
به این منظور کافی است از صفحه‌ی کارمندان کارمند مورد نظر انتخاب شود و در صفحه‌ی اطلاعات کارمند، گزینه‌ی  ردیاب کارمند (نماد پین نقشه) انتخاب شود. به این ترتیب در صفحه‌ی ردیاب کارمند کاری کافی است اطلاعات ردیاب را وارد کرده یا گزینه‌ی حذف را انتخاب کنیم.

\item \textbf{تولید گزارش مسیر‌های روزانه} \newline

به این منظور کافی است از صفحه‌ی کارمندان کارمند مورد نظر انتخاب شود و در صفحه‌ی اطلاعات کارمند، گزینه‌ی  گزارش‌های کارمند انتخاب شود. به این ترتیب در صفحه‌ی گزارش‌های کارمند کافی است گزینه‌ی گزارش مسیر‌های روزانه را انتخاب کنیم تا مسیر‌ها از طریق یک رابط نقشه به نمایش در‌آیند.

\item \textbf{تولید گزارش فعالیت‌های روزانه} \newline

به این منظور کافی است از صفحه‌ی کارمندان کارمند مورد نظر انتخاب شود و در صفحه‌ی اطلاعات کارمند، گزینه‌ی  گزارش‌های کارمند انتخاب شود. به این ترتیب در صفحه‌ی گزارش‌های کارمند کافی است گزینه‌ی گزارش فعالیت‌های روزانه را انتخاب کنیم تا فعالیت‌ها به صورت لیستی به نمایش در‌آیند.

\item \textbf{تولید گزارش ساعات مفید کاری} \newline

به این منظور کافی است از صفحه‌ی کارمندان کارمند مورد نظر انتخاب شود و در صفحه‌ی اطلاعات کارمند، گزینه‌ی  گزارش‌های کارمند انتخاب شود. به این ترتیب در صفحه‌ی گزارش‌های کارمند کافی است گزینه‌ی گزارش ساعات مفید کاری را انتخاب کنیم تا گزارش ساعات مفید کاری به صورت بصری شده به نمایش در آید.

\item \textbf{انجام حضور و غیاب} \newline
سامانه در ابتدای هر روز از طریق زیرسیستم پیام رسانی و در بخش پیام‌های درخواستی (request) حضور و غیاب کارمندان را انجام خواهد داد. به این ترتیب کارمندان کافی است پیام حضور و غیاب را تایید کنند.
\item \textbf{ثبت سفارش} \newline
برای ثبت سفارش کافی است از صفحه‌ی مدیریت سفارشات گزینه‌ی ثبت سفارش انتخاب شود. سپس کافی است در صفحه‌ی ورود اطلاعات سفارش، اطلاعات مورد نظر وارد شده و در صورت تمایل، کارمندی هم برای انجام سفارش تعیین شود.
\item \textbf{مشاهده‌ی وضعیت سفارش} \newline
برای مشاهده‌ی  وضعیت سفارش کافی است از صفحه‌ی مدیریت سفارشات گزینه‌ی سفارش مورد نظر انتخاب شود. سپس صفحه‌ی اطلاعات سفارش نشان داده خواهد شد که یکی از اطلاعات آن که به صورت بصری نمایش داده شده است، بیانگر وضعیت سفارش است.

\item \textbf{الصاق سفارش به کارمند} \newline
به این منظور کافی است حین ثبت سفارش کارمند مورد نظر انتخاب شود یا پس از ثبت سفارش با  مراجعه به بخش اطلاعات سفارش (از صفحه‌ی مدیریت سفارشات گزینه‌ی سفارش مورد نظر انتخاب شود. سپس صفحه‌ی اطلاعات سفارش نشان داده خواهد شد.)
کارمند مورد نظر را انتخاب شود.
\item \textbf{ویرایش سفارش (شامل حذف)} \newline
برای ویرایش سفارش  کافی است از صفحه‌ی مدیریت سفارشات گزینه‌ی سفارش مورد نظر انتخاب شود. سپس صفحه‌ی اطلاعات سفارش نشان داده خواهد شد که یکی از گزینه‌های آن ویرایش سفارش است. با انتخاب آن، صفحه‌ای مشابه صفحه‌ی ثبت سفارش نمایش داده خواهد شد که اطلاعات سفارش در فیلد‌های آن پر شده است. در این صفحه می‌توان تغییرات لازم را اعمال کرده و گزینه‌ی ذخیره‌سازی را انتخاب کرد.

\item \textbf{الصاق خودکار} \newline
برای الصاق خودکار سفارش به مناسب ترین کارمند کافی است از صفحه‌ی مدیریت سفارشات گزینه‌ی سفارش مورد نظر انتخاب شود. سپس صفحه‌ی اطلاعات سفارش نشان داده خواهد شد که یکی از گزینه‌های الصاق خودکار سفارش است.
\item \textbf{تغییر وضع سفارش} \newline
برای تغییر  وضعیت سفارش کافی است از صفحه‌ی سفارشات گزینه‌ی سفارش مورد نظر انتخاب شود. سپس صفحه‌ی اطلاعات سفارش نشان داده خواهد شد که یکی از اطلاعات آن که به صورت بصری نمایش داده شده است، بیانگر وضعیت سفارش است.
با انتخاب مرحله‌ی مورد نظر روی نمایش بصری ارائه شده، وضع سفارش به آن مرحله تغییر می‌کند.

\item \textbf{تولید گزارش هزینه‌های پنل} \newline
برای تولید گزارش هزینه‌های پنل کافی است از صفحه‌ی مدیریت مالی و گزارشات گزینه‌ی تولید گزارش هزینه‌های پنل انتخاب شود. به این ترتیب گزارش مورد نظر تولید شده و در صحفه‌ی دیگری که باز می‌شود نمایش داده خواهد شد.
\item \textbf{تولید گزارش سفارش‌های ثبت شده} \newline
برای تولید گزارش سفارش‌های ثبت شده کافی است از صفحه‌ی مدیریت مالی و گزارشات گزینه‌ی تولید گزارش سفارش‌های ثبت شده انتخاب شود. به این ترتیب گزارش مورد نظر تولید شده و در صحفه‌ی دیگری که باز می‌شود نمایش داده خواهد شد.
\item \textbf{تولید گزارش درآمد‌های پنل} \newline
برای تولید گزارش درآمد‌های پنل کافی است از صفحه‌ی مدیریت مالی و گزارشات گزینه‌ی تولید گزارش درآمد‌های پنل انتخاب شود. به این ترتیب گزارش مورد نظر تولید شده و در صحفه‌ی دیگری که باز می‌شود نمایش داده خواهد شد.
\item \textbf{تولید گزارش وظیفه‌های انجام شده} \newline
برای تولید گزارش وظیفه‌های انجام شده کافی است از صفحه‌ی مدیریت مالی و گزارشات گزینه‌ی تولید گزارش وظیفه‌های انجام شده انتخاب شود. به این ترتیب گزارش مورد نظر تولید شده و در صحفه‌ی دیگری که باز می‌شود نمایش داده خواهد شد.
\item \textbf{تولید گزارش ساعات مفید کاری} \newline
برای تولید گزارش ساعات مفید کاری کافی است از صفحه‌ی مدیریت مالی و گزارشات گزینه‌ی تولید گزارش ساعات مفید کاری انتخاب شود. به این ترتیب گزارش مورد نظر تولید شده و در صحفه‌ی دیگری که باز می‌شود نمایش داده خواهد شد.
\item \textbf{محاسبه‌ی دستمزد‌ها} \newline
برای محاسبه‌ی دستمزد‌ها کافی است از صفحه‌ی مدیریت مالی و گزارشات گزینه‌ی محاسبه‌ی دستمزد‌ها انتخاب شود. به این ترتیب دستمزد‌های کارمندان محاسبه شده و در صحفه‌ی دیگری که باز می‌شود نمایش داده خواهد شد.
\item \textbf{پیام به کارمند خاص} \newline
برای ارسال پیام به کارمند خاص کافی است از صفحه‌ی پیام رسانی و در تب پیام‌ها دکمه‌ی فعالیت قرمز رنگ انتخاب شده و از منوی باز شده گزینه‌ی پیام جدید انخاب شود. سپس در صفحه‌ی بعد پیام مورد نظر به همراه مخاطب آن تعیین و وارد شود. در نهایت با انتخاب گزینه‌ی ارسال، پیام ارسال خواهد شد.
\item \textbf{پیام به مدیریت} \newline

برای ارسال پیام به مدیریت کافی است از صفحه‌ی پیام رسانی و در تب درخواست‌ها دکمه‌ی فعالیت قرمز رنگ انتخاب شده و از منوی باز شده گزینه‌ی درخواست جدید انخاب شود. سپس در صفحه‌ی بعد، نوع درخواست، «other» انتخاب شده و متن مورد نظر وارد شود. در نهایت با انتخاب گزینه‌ی ارسال، پیام ارسال خواهد شد.
\item \textbf{ثبت اعلان عمومی} \newline
برای ثبت اعلان عمومی است از صفحه‌ی پیام رسانی و در تب اعلان‌ها دکمه‌ی فعالیت قرمز رنگ انتخاب شده و از منوی باز شده گزینه‌ی اعلان جدید انخاب شود. سپس در صفحه‌ی بعد پیام مورد نظر تعیین و وارد شود. در نهایت با انتخاب گزینه‌ی ارسال، اعلان ارسال خواهد شد.
\item \textbf{درخواست شیفت شناور} \newline
برای درخواست شیفت شناور کافی است از صفحه‌ی پیام رسانی و در تب درخواست‌ها دکمه‌ی فعالیت قرمز رنگ انتخاب شده و از منوی باز شده گزینه‌ی درخواست جدید انخاب شود. سپس در صفحه‌ی بعد، نوع درخواست، «floating-shift» انتخاب شده و متن مورد نظر وارد شود. در نهایت با انتخاب گزینه‌ی ارسال، پیام ارسال خواهد شد.
\item \textbf{درخواست مرخصی} \newline
برای درخواست مرخصی کافی است از صفحه‌ی پیام رسانی و در تب درخواست‌ها دکمه‌ی فعالیت قرمز رنگ انتخاب شده و از منوی باز شده گزینه‌ی درخواست جدید انخاب شود. سپس در صفحه‌ی بعد، نوع درخواست، «day-off» انتخاب شده و متن مورد نظر وارد شود. در نهایت با انتخاب گزینه‌ی ارسال، پیام ارسال خواهد شد.
\item \textbf{ویرایش اعلان (شامل حذف)} \newline
برای ویرایش یا حذف اعلان عمومی کافی است از صفحه‌ی پیام رسانی و در تب اعلان‌ها اعلان مورد نظر انتخاب شده و در صفحه‌ی مشاهده‌ی اعلان  گزینه‌ی ویرایش یا حذف (ARCHIVE) اعلان انخاب شود. در صورت ویرایش متن جدید نوشته شده و دکمه‌ی ثبت انتخاب می‌شود. اعلان ویرایش یا حذف خواهد شد.
\item \textbf{مشاهده‌ی پیام} \newline
برای مشاهده‌ی پیام دریافتی کافی است از صفحه‌ی پیام رسانی و در تب پیام‌ها پیام مورد نظر انتخاب شود. صفحه‌ی مشاهده‌ی پیام باز خواهد شد و پیام نمایش داده می‌شود.

\item \textbf{حذف پیام} \newline
برای حذف پیام دریافتی کافی است از صفحه‌ی پیام رسانی و در تب پیام‌ها پیام مورد نظر انتخاب شده و در صفحه‌ی مشاهده‌ی پیام  گزینه‌ی حذف پیام انخاب شود. پیام حذف خواهد شد.
\end{itemize}